% =============================================================================
% 04_price_formation.tex
% =============================================================================
% Sources: Harris Ch.13-14; Lehalle Ch.3.2-3.3; Cartea Ch.1
% =============================================================================

\subsection{Price Formation and Bid-Ask Spreads}

\begin{definition} \hlt{Price Formation Process (PFP)}\\
Mechanism by which markets balance supply and demand via the occurrence of deals between traders, forming prices that constitute a fair view of the value of exchanged assets.
\begin{enumerate}[label=\roman*.]
\setlength{\itemsep}{0pt}
\item Market Impact: trading pressure not consistent with market consensus generates temporary impact. When this is coherent with market, the impact is permanent.
\item Temporary Imbalances: Oscillating prices come from temporary imbalances between buyers and sellers that could be suppressed if investors were more synchronised.
\item Market-Makers' Role: Profit from temporary impact by buying from early sellers and selling to later buyers. Exposed to risk from unexpected news between arrival of sellers and buyers.
\end{enumerate}
\end{definition}

\begin{remark} \hlt{Information and Liquidity Paradox}\\
The more information available, the better the PFP. But trader fears information leakage and the threat of being front-run or having excessive market impact. Market design must balance to allow enough information sharing to ensure a fair PFP, while protecting each investor's interest from information leakage.
\end{remark}

\begin{remark} \hlt{Best Execution}\\
Best execution is not simply about immediate price improvement on aggressive orders. In fragmented markets, best execution requires:
\begin{enumerate}[label=\roman*.]
\setlength{\itemsep}{0pt}
\item Consolidated access to liquidity for both passive and aggressive orders.
\item Access to internal crossing engines to value ``natural liquidity'' from final investors.
\item Protection from information leakage and front-running by faster players.
\item Consideration of adverse selection, passive split, and reversal measurements over longer time scales.
\end{enumerate}
The concept of ``efficient execution'' can only be defined on a long time scale and must relate to the investment style of the order originator.
\end{remark}

\subsubsection{Price Discovery}
% How prices come to reflect information
% - Information aggregation
% - Price efficiency
% - Price discovery metrics (Hasbrouck, Gonzalo-Granger)
% - Contribution of different venues
% Sources: Harris Ch.10; Lehalle Ch.3.2; Johnson Ch.8

\begin{remark} \hlt{Fair Value vs. Market Price}\\
Fair value reflects asset's actual (intrinsic) value. Market price reflects what market is prepared to pay, incorporating expectations for future value. If demand is high and supply is limited, assets trade at a premium to fair value; if demand is low, discounting occurs. Fundamental analysts strive to determine fair value to assess whether trading is worthwhile. Technical analysts base pricing solely on market price and volume trends.
\end{remark}

\begin{remark} \hlt{Role of Market Transparency in Price Discovery}\\
A market maker's two-way quote provides only one view on pricing. If only best bid/offer from an order book is displayed, it is effectively the same as a two-way quote, adding uncertainty about available liquidity, which causes orders to be priced more aggressively than necessary. If quotes from multiple dealers or more of the order book are visible, traders can see the range of available prices and sizes. Visible liquidity allows traders to adjust their own valuations and determine target prices more accurately.
\end{remark}

\begin{remark} \hlt{Present Value Theory and Asset Pricing}\\
The discounted cash flow model states that present value of an asset corresponds to a discounted sum of its future payments. The discount rate accounts for time value of money and risk factors. Differing information and data means investors will achieve different prices, and this diversity is crucial to keep markets flowing.
\end{remark}

\subsubsection{The Bid-Ask Spread}
% Fundamental spread concepts
% - Quoted vs. effective spread
% - Realized spread
% - Price improvement
% - Half-spread conventions
% Sources: Harris Ch.14; Johnson Ch.2

\subsubsection{Spread Decomposition}
% Components of the bid-ask spread
% - Order processing costs
% - Inventory holding costs  
% - Adverse selection costs
% - Competition effects
% Empirical decomposition methods (Huang-Stoll, Glosten-Harris)
% Sources: Harris Ch.14; Lehalle Ch.2.3

\subsubsection{Tick Size and Discreteness}
% Effects of minimum price increments
% - Tick size regimes
% - Binding vs. non-binding constraints
% - Tick size and liquidity relationship
% - Sub-penny pricing
% Sources: Harris Ch.6; Lehalle Ch.1.3

\subsubsection{Price Impact Signatures}
% Empirical patterns in price movements
% - Trade sign autocorrelation
% - Price response to trades
% - Long-memory in order flow
% - Master curve of price impact
% Sources: Lehalle Ch.3.2; Cartea Ch.2
