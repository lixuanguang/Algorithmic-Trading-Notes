% =============================================================================
% 04_price_formation.tex
% =============================================================================
% Sources: Harris Ch.13-14; Lehalle Ch.4; Cartea Ch.1
% =============================================================================

\subsection{Price Formation and Bid-Ask Spreads}

\begin{definition} \hlt{Price Formation Process (PFP)}\\
Mechanism by which markets balance supply and demand via the occurrence of deals between traders, forming prices that constitute a fair view of the value of exchanged assets.
\begin{enumerate}[label=\roman*.]
\setlength{\itemsep}{0pt}
\item Market Impact: trading pressure not consistent with market consensus generates temporary impact. When this is coherent with market, the impact is permanent.
\item Temporary Imbalances: Oscillating prices come from temporary imbalances between buyers and sellers that could be suppressed if investors were more synchronised.
\item Market-Makers' Role: Profit from temporary impact by buying from early sellers and selling to later buyers. Exposed to risk from unexpected news between arrival of sellers and buyers.
\end{enumerate}
\end{definition}

\begin{remark} \hlt{Information and Liquidity Paradox}\\
The more information available, the better the PFP. But trader fears information leakage and the threat of being front-run or having excessive market impact. Market design must balance to allow enough information sharing to ensure a fair PFP, while protecting each investor's interest from information leakage.
\end{remark}

\begin{remark} \hlt{Best Execution}\\
Best execution is not simply about immediate price improvement on aggressive orders. In fragmented markets, best execution requires:
\begin{enumerate}[label=\roman*.]
\setlength{\itemsep}{0pt}
\item Consolidated access to liquidity for both passive and aggressive orders.
\item Access to internal crossing engines to value ``natural liquidity'' from final investors.
\item Protection from information leakage and front-running by faster players.
\item Consideration of adverse selection, passive split, and reversal measurements over longer time scales.
\end{enumerate}
The concept of ``efficient execution'' can only be defined on a long time scale and must relate to the investment style of the order originator.
\end{remark}

\subsubsection{Price Discovery}
% How prices come to reflect information
% - Information aggregation
% - Price efficiency
% - Price discovery metrics (Hasbrouck, Gonzalo-Granger)
% - Contribution of different venues
% Sources: Harris Ch.10; Lehalle Ch.4

\subsubsection{The Bid-Ask Spread}
% Fundamental spread concepts
% - Quoted vs. effective spread
% - Realized spread
% - Price improvement
% - Half-spread conventions
% Sources: Harris Ch.13; Johnson Ch.9

\subsubsection{Spread Decomposition}
% Components of the bid-ask spread
% - Order processing costs
% - Inventory holding costs  
% - Adverse selection costs
% - Competition effects
% Empirical decomposition methods (Huang-Stoll, Glosten-Harris)
% Sources: Harris Ch.14; Lehalle Ch.4

\subsubsection{Tick Size and Discreteness}
% Effects of minimum price increments
% - Tick size regimes
% - Binding vs. non-binding constraints
% - Tick size and liquidity relationship
% - Sub-penny pricing
% Sources: Harris Ch.6; Lehalle Ch.4

\subsubsection{Price Impact Signatures}
% Empirical patterns in price movements
% - Trade sign autocorrelation
% - Price response to trades
% - Long-memory in order flow
% - Master curve of price impact
% Sources: Lehalle Ch.4; Cartea Ch.2
