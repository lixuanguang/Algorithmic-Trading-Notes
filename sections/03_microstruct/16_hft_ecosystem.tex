% =============================================================================
% 16_hft_ecosystem.tex
% =============================================================================
% Sources: Lehalle Ch.12-13; Cartea Ch.3-4; Johnson Ch.17
% =============================================================================

\subsection{High-Frequency Trading Ecosystem}

\subsubsection{Defining High-Frequency Trading}
% Characteristics and taxonomy
% - Speed and latency
% - Holding periods
% - Order-to-trade ratios
% - Flat end-of-day positions
% - Capital efficiency
% Sources: Lehalle Ch.12; Johnson Ch.17

\subsubsection{HFT Strategies}
% Strategy categories
% - Market making (electronic)
% - Statistical arbitrage
% - Latency arbitrage
% - Structural arbitrage
% - Momentum ignition (controversial)
% Sources: Cartea Ch.3-4; Lehalle Ch.12

\subsubsection{Technology and Infrastructure}
% Speed and systems
% - Co-location and proximity hosting
% - Network optimization
% - FPGA and ASIC hardware
% - Software optimization
% - Market data feeds (direct vs. consolidated)
% Sources: Johnson Ch.17; Lehalle Ch.13

\subsubsection{Latency and the Speed Race}
% Competition for speed
% - Latency measurement
% - Latency arbitrage mechanics
% - Arms race dynamics
% - Speed bumps and deliberate delays
% Sources: Lehalle Ch.13; Cartea Ch.4

\begin{definition} \hlt{Nyquist-Shannon Sampling Theorem}\\
A signal can be perfectly reconstructed from its samples if the sampling frequency $f_s$  is at least twice the highest frequency component $f_{\max}$ in the signal:
\begin{equation*}
f_s \geq 2 f_{\max}
\end{equation*}
\end{definition}

\begin{remark} \hlt{Nyquist-Shannon in Market Microstructure}\\
The theorem has important implications for market data and trading:
\begin{enumerate}[label=\roman*.]
\setlength{\itemsep}{0pt}
\item Aliasing: If market data is sampled too slowly relative to the true dynamics, high-frequency patterns appear as spurious low-frequency patterns. Daily data cannot capture intraday mean reversion.
\item HFT Advantage: High-frequency traders sample orderbook states at microsecond intervals, capturing dynamics invisible to slower participants sampling at seconds or minutes.
\item Signature Plot: Realised volatility estimates depend on sampling frequency. Too high frequency introduces microstructure noise; too low misses true volatility. Nyquist frequency identify optimal sampling rate.
\item Epps Effect: Correlation between assets appears to decrease at higher sampling frequencies due to asynchronous trading, a manifestation of sampling issues in multivariate settings.
\item Practical Implication: To study phenomena at timescale $\tau$, one needs data sampled at least at frequency $1/(2\tau)$. Tick-by-tick data is necessary for microstructure research.
\end{enumerate}
\end{remark}

\subsubsection{HFT and Market Quality}
% Impact on market structure
% - Liquidity provision
% - Spread compression
% - Volatility effects
% - Flash crashes
% - Debates and controversies
% Sources: Lehalle Ch.12; Cartea Ch.3

\subsubsection{The HFT Business Model}
% Economics of HFT
% - Revenue sources
% - Cost structure
% - Barriers to entry
% - Consolidation trends
% - Regulatory pressures
% Sources: Lehalle Ch.12-13
