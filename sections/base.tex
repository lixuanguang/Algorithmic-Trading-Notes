% -----------------------------
% Document Class & Encoding
% -----------------------------
\usepackage[utf8]{inputenc}  % Input encoding
\usepackage[T1]{fontenc}     % Font encoding

% -----------------------------
% Page Layout & Geometry
% -----------------------------
\usepackage[margin=2cm,foot=1cm]{geometry}

% -----------------------------
% Header & Footer Management
% -----------------------------
\usepackage{fancyhdr}
\pagestyle{fancy}

% -----------------------------
% Text Formatting & Spacing
% -----------------------------
\usepackage{ragged2e}
\usepackage[nodisplayskipstretch]{setspace}
\setstretch{1}  % Line spacing

% -----------------------------
% Hyperlinks
% -----------------------------
\usepackage{hyperref}
\hypersetup{
  colorlinks,
  linkcolor={blue!80!black},
  citecolor={green!80!black},
  urlcolor={red!80!black}
}

% -----------------------------
% Bibliography Support
% -----------------------------
\usepackage{natbib}

% -----------------------------
% Listings for Code
% -----------------------------
\usepackage{listings}
\usepackage{xcolor}

% -----------------------------
% Tables, Figures & Graphics
% -----------------------------
\usepackage{tabularx}
\usepackage{multirow}
\usepackage{pdflscape}
\usepackage{graphicx}

% -----------------------------
% Algorithms
% -----------------------------
\usepackage{algorithm}       % For floating algorithm environment
\usepackage{algpseudocode}   % For pseudo-code inside algorithms

% -----------------------------
% Math Packages
% -----------------------------
\usepackage{amsmath,amssymb,amsthm,mathtools,bm}
\usepackage{mathrsfs}        % Script fonts
\usepackage{tikz-cd}         % Commutative diagrams
\usepackage{rotating}        % Rotated objects
\usepackage{float}           % Float placement
\usetikzlibrary{calc, arrows, positioning, arrows.meta}

% -----------------------------
% TikZ and its Libraries
% -----------------------------
\usepackage{tikz}
\usetikzlibrary{positioning} % Added to support "below=of" syntax
\usepackage{forest}          % For advanced tree diagrams
\usepackage{tikz-qtree}      % For vertical tree diagrams

% -----------------------------
% Miscellaneous Packages
% -----------------------------
\usepackage{xparse}          % For defining robust commands
\usepackage{enumitem}        % For customizing list layouts
\usepackage{makecell}        % For enhanced table cells
\usepackage{comment}		 % For commenting out sections

%%%%%%%%%%%%%%%%%%%%%%%%%%%%%%%%%%%%%%%%%%%%%%%%%%%%%%%%%%%%%%%%%%%%%%%%%%%%%%%
% Custom Math Commands and Shortcuts
%%%%%%%%%%%%%%%%%%%%%%%%%%%%%%%%%%%%%%%%%%%%%%%%%%%%%%%%%%%%%%%%%%%%%%%%%%%%%%%

% Matrix and Vector Formatting
\newcommand{\matr}[1]{\mathbf{#1}}      % For undergraduate algebra (alternative: \bm{#1})
\newcommand{\vect}[1]{\bm{#1}}

% Standard Number Sets
\newcommand{\N}{\mathbb{N}}
\newcommand{\R}{\mathbb{R}}
\newcommand{\Q}{\mathbb{Q}}
\newcommand{\C}{\mathbb{C}}
\newcommand{\Z}{\mathbb{Z}}

% Bold variables (customize as needed)
\newcommand{\x}{\mathbf{x}}
\newcommand{\F}{\mathbf{F}}
\newcommand{\f}{\mathbf{f}}
\newcommand{\y}{\mathbf{y}}
\renewcommand{\a}{\mathbf{a}}
\renewcommand{\b}{\mathbf{b}}
\renewcommand{\c}{\mathbf{c}}
\newcommand{\h}{\mathbf{h}}
\newcommand{\g}{\mathbf{g}}
\newcommand{\z}{\mathbf{z}}
\newcommand{\ze}{\mathbf{0}}

% Norms, Absolute Values, and Bracketing
\newcommand{\norm}[1]{\left\lVert#1\right\rVert}
\newcommand{\abs}[1]{\left\lvert#1\right\rvert}
\newcommand{\brk}[1]{\left[#1\right]}
\newcommand{\brc}[1]{\left\{#1\right\}}
\newcommand{\paren}[1]{\left(#1\right)}
\newcommand{\normop}[1]{\left\lVert#1\right\rVert_\text{op}}

% Miscellaneous Math Shortcuts
\newcommand{\LL}{\mathcal{L}}
\newcommand{\uni}{\overset{\text{uni}}{\to}}
\DeclareMathOperator{\diam}{diam}
\newcommand{\Prr}[1]{\text{Pr}\left(#1\right)}
\newcommand{\ceil}[1]{\lceil #1 \rceil}
\newcommand{\floor}[1]{\lfloor #1 \rfloor}
\newcommand\disteq{\stackrel{\textnormal{dist}}{=}}

% Evaluation command with options
\NewDocumentCommand{\evalat}{sO{\big}mm}{%
  \IfBooleanTF{#1}
    {\mleft. #3 \mright|_{#4}}
    {#3#2|_{#4}}%
}

%%%%%%%%%%%%%%%%%%%%%%%%%%%%%%%%%%%%%%%%%%%%%%%%%%%%%%%%%%%%%%%%%%%%%%%%%%%%%%%
% Color Definitions
%%%%%%%%%%%%%%%%%%%%%%%%%%%%%%%%%%%%%%%%%%%%%%%%%%%%%%%%%%%%%%%%%%%%%%%%%%%%%%%

\definecolor{dartmouthgreen}{rgb}{0.05, 0.5, 0.06}
\definecolor{egyptianblue}{rgb}{0.06, 0.2, 0.65}
\definecolor{dukeblue}{rgb}{0.0, 0.0, 0.61}
\definecolor{jazzberryjam}{rgb}{0.65, 0.04, 0.37}
\definecolor{magenta}{HTML}{EC008C}
\definecolor{darkmagenta}{rgb}{0.55, 0.0, 0.55}
\definecolor{deeppink}{rgb}{1.0, 0.08, 0.58}
\definecolor{codegreen}{rgb}{0,0.6,0}
\definecolor{codegray}{rgb}{0.5,0.5,0.5}
\definecolor{codepurple}{rgb}{0.58,0,0.82}
\definecolor{backcolour}{rgb}{0.95,0.95,0.92}

%%%%%%%%%%%%%%%%%%%%%%%%%%%%%%%%%%%%%%%%%%%%%%%%%%%%%%%%%%%%%%%%%%%%%%%%%%%%%%%
% Custom Shortcuts and Emphasis Commands
%%%%%%%%%%%%%%%%%%%%%%%%%%%%%%%%%%%%%%%%%%%%%%%%%%%%%%%%%%%%%%%%%%%%%%%%%%%%%%%

\newcommand{\isomorp}{\xrightarrow{\sim}}
\newcommand{\hlt}[1]{\textit{{\color{blue}#1}}}
\newcommand{\simpt}[1]{\textit{{\color{deeppink}#1}}}
\newcommand{\impt}[1]{\textit{{\color{red}#1}}}
\newcommand{\gcds}[1]{\textnormal{gcd}#1}
\newcommand{\mins}[1]{\textnormal{min}#1}
\newcommand{\maxs}[1]{\textnormal{max}#1}
\newcommand{\lcms}[1]{\textnormal{lcm}#1}
\newcommand{\degs}[1]{\textnormal{deg}#1}
\newcommand{\exps}[1]{\textnormal{exp}#1}
\newcommand{\tors}[1]{\textnormal{Tor}#1}
\newcommand{\homs}[1]{\textnormal{Hom}#1}
\newcommand{\anns}[1]{\textnormal{Ann}#1}
\DeclareMathOperator{\sign}{sign}

%%%%%%%%%%%%%%%%%%%%%%%%%%%%%%%%%%%%%%%%%%%%%%%%%%%%%%%%%%%%%%%%%%%%%%%%%%%%%%%
% Theorem and Definition Environments
%%%%%%%%%%%%%%%%%%%%%%%%%%%%%%%%%%%%%%%%%%%%%%%%%%%%%%%%%%%%%%%%%%%%%%%%%%%%%%%
\theoremstyle{plain}
\newtheorem{theorem}{Theorem}[subsection]
\newtheorem{corollary}[theorem]{Corollary}
\newtheorem{lemma}[theorem]{Lemma}
\newtheorem{proposition}[theorem]{Proposition}
\newtheorem{qnbank}[theorem]{Question Bank}
\let\oldqnbank\qnbank
\renewcommand{\qnbank}{\oldqnbank\normalfont}
\newtheorem{process}[theorem]{Process}
\let\oldprocess\process
\renewcommand{\process}{\oldprocess\normalfont}
\newtheorem{method}[theorem]{Method}
\let\oldmethod\method
\renewcommand{\method}{\oldmethod\normalfont}
\newtheorem{definition}[theorem]{Definition}
\let\olddefinition\definition
\renewcommand{\definition}{\olddefinition\normalfont}
\newtheorem{example}[theorem]{Example}
\let\oldexample\example
\renewcommand{\example}{\oldexample\normalfont}
\newtheorem{remark}[theorem]{Remark}
\let\oldremark\remark
\renewcommand{\remark}{\oldremark\normalfont}

%%%%%%%%%%%%%%%%%%%%%%%%%%%%%%%%%%%%%%%%%%%%%%%%%%%%%%%%%%%%%%%%%%%%%%%%%%%%%%%
% Breakable Algorithm Environment (Algorithms spanning multiple pages)
%%%%%%%%%%%%%%%%%%%%%%%%%%%%%%%%%%%%%%%%%%%%%%%%%%%%%%%%%%%%%%%%%%%%%%%%%%%%%%%

\makeatletter
\newenvironment{breakablealgorithm}
 {%
  \begin{center}
    \refstepcounter{algorithm} % Increment algorithm counter
    \hrule height.8pt depth0pt \kern2pt
    \renewcommand{\caption}[2][\relax]{%
      {\raggedright\textbf{Algorithm \thealgorithm:} ##2\par}%
      \ifx\relax##1\relax
        \addcontentsline{loa}{algorithm}{\protect\numberline{\thealgorithm}##2}%
      \else
        \addcontentsline{loa}{algorithm}{\protect\numberline{\thealgorithm}##1}%
      \fi
      \kern2pt\hrule\kern2pt
    }%
 }{%
    \kern2pt\hrule
  \end{center}
 }
\makeatother

%%%%%%%%%%%%%%%%%%%%%%%%%%%%%%%%%%%%%%%%%%%%%%%%%%%%%%%%%%%%%%%%%%%%%%%%%%%%%%%
% Code Listings Style
%%%%%%%%%%%%%%%%%%%%%%%%%%%%%%%%%%%%%%%%%%%%%%%%%%%%%%%%%%%%%%%%%%%%%%%%%%%%%%%

\lstdefinestyle{codestyle}{
  commentstyle=\color{codegreen},
  keywordstyle=\color{magenta},
  numberstyle=\tiny\color{codegray},
  stringstyle=\color{codepurple},
  basicstyle=\ttfamily\normalsize,
  breakatwhitespace=false,
  breaklines=true,
  captionpos=b,
  keepspaces=true,
  numbers=left,
  numbersep=10pt,
  frame=single,
  showspaces=false,
  showstringspaces=false,
  showtabs=false,
  tabsize=2
}
\lstset{style=codestyle}


