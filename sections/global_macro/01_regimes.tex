\subsection{Regime and World State}

\subsubsection{Survey of Global Macro Landscape}

\begin{definition} \hlt{Global Macro}\\
Global macro is a top-down investment strategy that trades economic regimes, not securities.\\
The core idea is simple but powerful: macro forces dominate asset prices when they shift, and those forces operate across countries, asset classes, and policy domains.\\
Global macro traders
\begin{enumerate}[label=\roman*.]
\setlength{\itemsep}{0pt}
\item Form views on growth, inflation, monetary policy, and geopolitics
\item Express those views via equities, rates, FX, and commodities
\item Are unconstrained by geography, asset class, or direction (long/short)
\end{enumerate}
Global macro distinguishes from other strategies due to flexibility to trade any market, any instrument, any direction, and horizon. Hence macro strategies survive regime change while narrower strategies do not.
\end{definition}

\begin{remark} \hlt{Existence of Global Macro as a Strategy}\\
Macro exists because economic regimes are unstable, but markets price them slowly and asymmetrically.\\
These is due to three structural reasons:
\begin{enumerate}[label=\roman*.]
\setlength{\itemsep}{0pt}
\item Policy is discrete, markets are continuous. Central banks and governments move in jumps (with rate decisions, QE, capital controls, etc.). Markets must reprice entire distributions after the fact.
\item Capital is sticky. Large allocators (pensions, insurers, sovereign funds) cannot reposition quickly. Macro traders exploit this inertia.
\item Cross-asset linkages are under-arbitraged. Many investors specialise in certain asset class. Macro traders operate between silos.
\end{enumerate}
Hence macro is a strategy of structural inefficiency, not informational advantage.
\end{remark}

\begin{remark} \hlt{Types of Global Macro Strategies}\\
The four board approaches are:
\begin{enumerate}[label=\roman*.]
\setlength{\itemsep}{0pt}
\item Discretionary macro.\\
Requires human judgment, narrative synthesis, pattern recognition. Strong in crises and regime breaks
\item Systematic macro.\\
Model-driven, factor-based, diversified. Strong over long horizons, weaker in abrupt transitions.
\item High-frequency macro.\\
Focuses on short-term dislocations around macro events. Limited scalability
\item CTAs (trend-followers).\\
Primarily futures, momentum-driven. Strong over full cycles, vulnerable to choppy markets
\end{enumerate}
\end{remark}

\begin{remark} \hlt{Return Profile Across Cycles}\\
Global macro’s defining feature is return asymmetry across market regimes.\\
Empirically, global macro has lower average volatility than equities, comparable or higher long-run returns, and ow or negative correlation to equities during crises.\\
Macro tends to underperform in bull markets, but outperform during dislocations, recessions, and policy shifts.\\
This is because macro traders can go short, can hold cash, can trade volatility, rates, and FX directly.
\end{remark}

\begin{remark} \hlt{Behaviour in Economic Crisis}\\
In stress regimes, equity managers are trapped long, credit liquidity vanishes, correlations converge toward 1.\\
Macro funds, by contrast, can reduce gross exposure, can short risk assets, can express views via rates, FX, or volatility instead of equities.
\end{remark}

\begin{remark} \hlt{Global Macro Correlation Properties}\\
Global macro’s real value is conditional correlation. \\
In normal times, there is low correlation. In crises, there is negative correlation to equities.
\end{remark}

\subsubsection{Role of Central Banks in Global Macro}

\begin{remark} \hlt{Monetary Policy Regimes}\\
Central banks define the macro regime more than any actor. Markets do not trade rates, but trade policy intent under constraints. Regimes fall into following types:
\begin{enumerate}[label=\roman*.]
\setlength{\itemsep}{0pt}
\item \hlt{Conventional Policy Regime}: Policy rate is the main tool. Yield curve responds normally. Transmission is through rates to credit, to growth, then to inflation.
\item \hlt{Unconventional Policy Regime}: Policy rate loses effectiveness. Balance sheet becomes the primary tool. Transmission shifts to asset prices, expectations, and FX.
\item \hlt{Crisis / Emergency Regime}: Stability is more important than inflation targeting. Rules are suspended. Liquidity provision dominates all other objectives.
\end{enumerate}
\end{remark}

\begin{remark} \hlt{Objectives of Central Banks}\\
Central banks aim for price stability (inflation control), full employment and growth stability, financial system stability, and currency stability. Under stress, these objectives may conflict, and central banks must choose one criteria to sacrifice.
\end{remark}

\begin{remark} \hlt{Conventional Central Bank Tools}
\begin{enumerate}[label=\roman*.]
\setlength{\itemsep}{0pt}
\item \hlt{Reserve Ratio}: closely related to monetary policy. Raising reserve ratio decreases money supply.
\item \hlt{Interest Rate}: set by changing overnight rate at which banks lend and borrow to meet reserve requirements.
\item \hlt{Open Market Operations}: process of buying and selling government bonds in local currency. Repurchase agreements (repos) are shorted term agreements in which the bank agrees to repurchase securities purchased by the Fed within a specified amount of time.
\item \hlt{Currency Intervention}: control inflation and demand for imports and exports
\begin{enumerate}[label=\arabic*.]
\setlength{\itemsep}{0pt}
\item Sterilised Intervention: trading own currency via foreign exchange reserves.
\item Unsterilised Intervention: allows for changes in monetary base
\end{enumerate}
\end{enumerate}
\end{remark}

\begin{remark} \hlt{Taylor's Rule}\\
Maps macroeconomic deviations to policy stance. The policy interest rate is stated as
\begin{equation*}
i_t = r^* + \pi_t + \alpha (\pi_t - \pi^*) + \beta (y_t - y_t^*) \nonumber
\end{equation*}
where $r^*$ is the neutral real rate (unobservable), $\pi_t$ is the current inflation, $\pi^*$ is the inflation target, $(y_t - y_t^*)$ is the actual vs potential output, $\alpha, \beta$ are the response coefficients (usually $0.5$).\\
The rule states that the nominal policy rate must rise more than one-to-one with inflation; otherwise real rates fall, inflation becomes self-reinforcing, and expectations de-anchor. \\
Central banks treat Taylor's rule as a benchmark, communication tool, and political shield.
\end{remark}

\begin{remark} \hlt{The Impossible Trinity}\\
A country cannot simultaneously have
\begin{enumerate}[label=\roman*.]
\setlength{\itemsep}{0pt}
\item Fixed exchange rate
\item Free capital flows
\item Independent monetary policy
\end{enumerate}
This constraint explains currency crisis, forced rate hikes or cuts, FX pegs breaking, and why EM central banks behave "irrationally" under stress.
\end{remark}

\begin{remark} \hlt{Central Bank Balance Sheet}
\begin{enumerate}[label=\roman*.]
\setlength{\itemsep}{0pt}
\item Monetary Base: sum of monetary liabilities, currency in circulation, and reserves. Can only be controlled with open market operations.
\item Money Supply: changes in supply are best predictors of change in GDP, growth, and inflation. Quantity Theory of Money is used to show effect of change in money supply.
\begin{equation*}
M \times V = P \times Q
\end{equation*}
where $M$ is money supply, $V$ is velocity of money (turnover rate), $P$ is price level, $Y$ is real output.
\end{enumerate}
\end{remark}

\begin{remark} \hlt{Types of Money Supply}
\begin{enumerate}[label=\roman*.]
\setlength{\itemsep}{0pt}
\item $M1$: Currency, traveler's checks; demand deposits; now and similar interest-earning checking accounts
\item $M2$: $M1$; savings deposits, money market (mm) deposit accounts; small time deposits; retail money market mutual fund balances
\item $M3$: $M2$; large time deposits; institutional mm mutual fund balances; repurchase agreements; eurodollars
\end{enumerate}
\end{remark}

\begin{remark} \hlt{Zero Lower Bound and the Liquidity Trap}\\
Central bank rate is at or near zero. Rate cuts lose marginal effectiveness; expectations dominate outcomes; forward guidance becomes critical; fiscal policy starts to matter more.\\
Yield curves flatten naturally; carry trades distort; volatility is suppressed.
\end{remark}

\begin{remark} \hlt{Quantitative Easing (QE)}\\
Central banks purchase financial assets such as domestic government debt, then create electronic reserves to increase money supply. Bank lending does not mechanically follow reserve creation.\\
QE works through portfolio rebalancing, risk premia compression, signalling effects.
\end{remark}

\begin{remark} \hlt{Central Bank Communication}\\
Gives perspective on how each central bank views its own economy, the global economy, and other factors which may influence monetary policy. Markets trade on hawkish vs dovish tone; changes in reaction function; credibility vs desperation; internal disagreement.
\end{remark}

\subsubsection{Economic Data Releases and Demographics}

\begin{remark} \hlt{Growth Indicators}\\
Growth data sets the directional bias for risk assets, but markets trade changes, not levels.
\begin{enumerate}[label=\roman*.]
\setlength{\itemsep}{0pt}
\item GDP (Headline, QoQ, YoY): measures total production in current or constant prices (nominal, real GDP).
\item Purchasing Managers' Indices (PMIs): surveys business managers on employment, new orders, inventories, production, supplier deliveries, customer inventories, prices, backlog of orders, imports, and exports on a scale of better, same, or worse than a month ago.
\item Industrial Production: measures growth in the manufacturing sector, utility companies, and the mining industry. More sensitive to economic cycles than services. Leading indicator for GDP forecasting.
\item Trade Volumes, Baltic Dry Index (BDI): shipping and trade index that tracks the cost of transporting raw materials. Leading economic indicator on global trade activity.
\end{enumerate}
PMIs and surveys move markets more than GDP, because they are higher frequency and more forward-looking.\\
Growth surprises affect equities (earnings expectations), commodities (demand), FX (via growth differentials).
\end{remark}

\begin{remark} \hlt{Inflation}\\
Inflation data is primarily a policy variable, not just a macro statistic.
\begin{enumerate}[label=\roman*.]
\setlength{\itemsep}{0pt}
\item Consumer Price Index (CPI)/Core CPI: prices paid by consumers for a basket of goods and services.
\item Producer Price Index (PPI): measures the average change over time in the selling prices received by domestic producers for their output.
\item Inflation Expectations (breakevens, surveys): rate at which consumers, businesses, investors expect prices to rise in the future.
\end{enumerate}
Inflation surprises directly affect real yields, curve shape, FX via real rate differentials. \\
Inflation matters most when central banks are near constraints (e.g. ZLB), credibility is questioned.\\
In low-inflation regimes, growth dominates. In high-inflation regimes, inflation dominates everything.
\end{remark}

\begin{remark} \hlt{Employment and Population}\\
Labor data is the bridge between growth, inflation, and central bank reaction.
\begin{enumerate}[label=\roman*.]
\setlength{\itemsep}{0pt}
\item Non Farm Payroll Number: measures the number of workers in the United States except those who work in farming, private households, nonprofits, and sole proprietorships or self-employment, as well as those who are active military service members
\item Unemployment Rate: number of unemployed as a percentage of the labor force
\item Participation Rate: number of people in the labor force as a percentage of the civilian non-institutional population that is either working or actively looking for work.
\item Wage Growth: yearly change in wages and salaries disbursements from government, manufacturing and service industries.
\end{enumerate}
Payrolls are among the highest-volatility releases across rates, FX, equities.\\
Wage growth is often more important than job count.\\
Demographics constrain labor markets tightness, affecting inflation persistence, long-term growth potential. 
\end{remark}

\begin{remark} \hlt{Balance of Payments}\\
External balances define structural currency pressure.
\begin{enumerate}[label=\roman*.]
\setlength{\itemsep}{0pt}
\item Current Account: change in net foreign assets, composing of net exports (exports $-$ imports), net foreign investment (foreign income $-$ foreign payments), and transfer payments.
\item Capital Account: calculated by subtracting the capital transfer payments and other debits from the capital transfer receipts and other credits
\item FX Reserves: cash and other reserve assets (i.e. gold, silver) held by central bank
\end{enumerate}
Persistent current-account deficits require capital inflows, higher yields, or currency depreciation.\\
Surplus countries tend to have structurally stronger currencies, reserve accumulation.
\end{remark}

\begin{remark} \hlt{Government Indicators}\\
Fiscal data determines sustainability, not short-term price action.
\begin{enumerate}[label=\roman*.]
\setlength{\itemsep}{0pt}
\item Budget Balance: amount spent minus amount received in revenue
\item Debt and Debt-to-GDP: measurement of how much a government borrows. The higher the debt-to-GDP ratio, the more it signifies an unhealthy economy.
\item Government Spending: expenditures provide a proxy for how much the government spends
\end{enumerate}
Fiscal deterioration matters when markets doubt financing ability, central banks lose independence.\\
Sovereign risk shows up first in bond spreads, CDS, FX (especially EM).
\end{remark}

\begin{remark} \hlt{Consumption Indicators}\\
Consumption is the engine of developed-market growth, especially the US.
\begin{enumerate}[label=\roman*.]
\setlength{\itemsep}{0pt}
\item Retail Sales: tracks consumer demand for finished goods, measured by durable and non-durable goods purchased over a defined period of time.
\item Consumer Confidence: measures what consumers are feeling about their expected financial situation
\item Personal Income: all income collectively received by all individuals or households in a country.
\end{enumerate}
Consumption data is most relevant for equity indices, cyclical sectors, domestic currencies.\\
Confidence often leads spending, but is noisy.
\end{remark}

\begin{remark} \hlt{Industry and Services Indicators}\\
These indicators provide information on where growth is coming from.
\begin{enumerate}[label=\roman*.]
\setlength{\itemsep}{0pt}
\item Manufacturing, Service PMIs
\item New Orders, Inventories, Backlogs
\end{enumerate}
Manufacturing is more cyclical and globally linked.\\
Services dominate GDP in developed markets but move slower.\\
Inventory cycles matter for commodities and industrial equities.
\end{remark}

\begin{remark} \hlt{Demographics}\\
Slowest-moving macro variable, but the hardest constraint.
\begin{enumerate}[label=\roman*.]
\setlength{\itemsep}{0pt}
\item Aging Populations
\item Labour Force Growth
\item Dependency Ratios: portion of the population that is not in the labor force.
\end{enumerate}
Demographics affect long-term growth, savings rates, current accounts, equilibrium real interest rates. \\
Explains Japan low yields, surplus countries accumulate reserves, some growth paths are structurally capped.
\end{remark}


