\subsection{Regime and World State}

\subsubsection{Survey of Global Macro Landscape}

\begin{definition} \hlt{Global Macro}\\
Global macro is a top-down investment strategy that trades economic regimes, not securities.\\
The core idea is simple but powerful: macro forces dominate asset prices when they shift, and those forces operate across countries, asset classes, and policy domains. Global macro traders
\begin{enumerate}[label=\roman*.]
\setlength{\itemsep}{0pt}
\item Form views on growth, inflation, monetary policy, and geopolitics
\item Express those views via equities, rates, FX, and commodities
\item Are unconstrained by geography, asset class, or direction (long/short)
\end{enumerate}
Global macro is often considered the most flexible and opportunistic hedge fund strategy, due to the scope of traded products and the number of markets it covers. Its aim is to preserve capital, using stringent risk management to limit drawdowns.
\end{definition}

\begin{remark} \hlt{Existence of Global Macro as a Strategy}\\
Macro exists because economic regimes are unstable, but markets price them slowly and asymmetrically.\\
Three structural reasons:
\begin{enumerate}[label=\roman*.]
\setlength{\itemsep}{0pt}
\item \textbf{Policy is discrete, markets are continuous}: Central banks and governments move in jumps (rate decisions, QE, capital controls). Markets must reprice entire distributions after the fact.
\item \textbf{Capital is sticky}: Large allocators (pensions, insurers, sovereign funds) cannot reposition quickly. Macro traders exploit this inertia.
\item \textbf{Cross-asset linkages are under-arbitraged}: Many investors specialise in a single asset class. Macro traders operate between silos.
\end{enumerate}
Hence macro is a strategy of structural inefficiency, not informational advantage.
\end{remark}

\begin{remark} \hlt{Types of Global Macro Strategies}\\
The four broad approaches are:
\begin{enumerate}[label=\roman*.]
\setlength{\itemsep}{0pt}
\item \textbf{Discretionary Macro}\\
Relies on trader experience, intelligence, and knowledge to take subjective bets on global markets. Requires serious organization and data processing skills. Uses top-down analysis of risks and opportunities across industries, sectors, countries, and the macroeconomic situation.\\
Discretionary traders can execute \textit{directional trades} (betting on asset direction) or \textit{relative value trades} (pairing assets to capture value differentials). Strong in crises and regime breaks where human judgment outperforms models.
\item \textbf{Systematic Macro}\\
Model-driven, employing quantitative factors to produce trading positions that remove human emotion. Prides itself on stringent process, strong back-tests, and ability to operate solely on quantitative analysis.\\
Over long periods, systematic funds produce more consistent returns than discretionary strategies; however, in periods of high volatility, they tend to underperform (e.g., 2008).\\
The ability to trade multiple liquid asset classes allows these funds to scale to very large AUM.
\item \textbf{High-Frequency Macro}\\
Uses sophisticated technology to trade very short-term (millisecond to hours) dislocations around macro events. Processing speed is paramount. Limited scalability compared to discretionary and systematic.
\item \textbf{CTAs (Commodity Trading Advisors)}\\
Primarily futures-based, momentum-driven trend-following strategies. Position sizing methodology originated with the Turtle Traders. Perform well over longer periods but subject to large drawdowns.
\end{enumerate}
It is wise to allocate to both discretionary and systematic macro in a balanced manner. Discretionary is negatively correlated during stress, while systematic provides predictable allocation with back-tested drawdowns.
\end{remark}

\begin{remark} \hlt{Trade Types in Macro}\\
Discretionary macro traders express views through two primary trade structures:
\begin{enumerate}[label=\roman*.]
\setlength{\itemsep}{0pt}
\item \textbf{Directional Trades}: Betting on an asset moving in a particular direction. Example: going long copper on a bullish commodities view.
\item \textbf{Relative Value Trades}: Pairing or grouping assets to capture the relative value differential between them. Example: during European crisis, short Italian 5-year BTPs vs.\ long German Bobls, betting Italy will see yields rise relative to Germany.
\item \textbf{Thematic Trades}: Longer-term positions based on structural macro views.\\
Examples: "fiat money debasement", hence long gold; "Japan debt unsustainability", hence short JGBs.\\
Thematic trades should be sized smaller than short-term trades because they can take years to play out and may incur negative carry.
\end{enumerate}
\end{remark}

\begin{remark} \hlt{Return Profile Across Cycles}\\
Global macro's defining feature is return asymmetry across market regimes. Empirical characteristics:
\begin{enumerate}[label=\roman*.]
\setlength{\itemsep}{0pt}
\item Lower average volatility than equities
\item Comparable or higher long-run returns
\item Low or negative correlation to equities during crises
\item Higher Sharpe ratio than most hedge fund strategies
\end{enumerate}
Macro tends to underperform in bull markets (when staying long works), but outperform during dislocations, recessions, and policy shifts. This asymmetry exists because macro traders can go short, can hold cash, can trade volatility, rates, and FX directly (flexibility that equity-focused strategies lack).
\end{remark}

\begin{remark} \hlt{Behaviour in Economic Crisis}\\
In stress regimes:
\begin{enumerate}[label=\roman*.]
\setlength{\itemsep}{0pt}
\item Equity managers are trapped long (mandates prevent shorting or holding cash)
\item Credit liquidity vanishes (bids disappear, spreads gap wider)
\item Correlations converge toward 1 (diversification fails)
\end{enumerate}
Macro funds, by contrast, can:
\begin{enumerate}[label=\roman*.]
\setlength{\itemsep}{0pt}
\item Reduce gross exposure rapidly
\item Short risk assets directly
\item Express views via rates, FX, or volatility instead of equities
\item Profit from the dislocation rather than merely survive it
\end{enumerate}
\end{remark}

\begin{remark} \hlt{Global Macro Correlation Properties}\\
Global macro's real value is conditional correlation:
\begin{enumerate}[label=\roman*.]
\setlength{\itemsep}{0pt}
\item In normal times: low correlation to equities
\item In crises: negative correlation to equities
\end{enumerate}
This makes macro an effective portfolio diversifier and tail-risk hedge. Combined with high liquidity (trading the most liquid markets globally), macro is the most popular hedge fund allocation by pension funds.
\end{remark}

\begin{remark} \hlt{Liquidity Advantage}\\
One of most liquid hedge fund strategies because it trades the most liquid underlying markets:
\begin{enumerate}[label=\roman*.]
\setlength{\itemsep}{0pt}
\item G10 FX and major EM currencies
\item Government bond futures (Treasuries, Bunds, JGBs, Gilts)
\item Equity index futures (S\&P $500$, Euro Stoxx, Nikkei, etc.)
\item Major commodity futures (crude oil, gold, copper, etc.)
\end{enumerate}
This liquidity allows macro funds to:
\begin{enumerate}[label=\roman*.]
\setlength{\itemsep}{0pt}
\item Scale to very large AUM without market impact
\item Enter and exit positions quickly during stress
\item Offer better redemption terms than illiquid strategies
\end{enumerate}
\end{remark}

\subsubsection{Role of Central Banks in Global Macro}

\begin{remark} \hlt{Monetary Policy Regimes}\\
Central banks define the macro regime more than any other actor. Markets do not trade rates, they trade policy intent under constraints. Regimes fall into following types:
\begin{enumerate}[label=\roman*.]
\setlength{\itemsep}{0pt}
\item \textbf{Conventional Policy Regime}: Policy rate is the main tool. Yield curve responds normally. Transmission is through rates to credit, to growth, then to inflation.
\item \textbf{Unconventional Policy Regime}: Policy rate loses effectiveness (zero lower bound). Balance sheet becomes the primary tool. Transmission shifts to asset prices, expectations, and FX.
\item \textbf{Crisis/Emergency Regime}: Stability is more important than inflation targeting. Rules are suspended. Liquidity provision dominates all other objectives.
\end{enumerate}
\end{remark}

\begin{remark} \hlt{Objectives of Central Banks}\\
Central banks typically have mandates that include:
\begin{enumerate}[label=\roman*.]
\setlength{\itemsep}{0pt}
\item \textbf{Price Stability}: Most central banks target inflation around $2\%$
\item \textbf{Full Employment}: Explicit in Fed's dual mandate; implicit elsewhere
\item \textbf{Financial System Stability}: Lender of last resort function; systemic risk monitoring
\item \textbf{Currency Stability}: Explicit for some (SNB, PBOC); implicit for others
\end{enumerate}
Under stress, these objectives may conflict. Central banks must choose which objective to sacrifice. Understanding these trade-offs is critical for macro trading.
\end{remark}

\begin{remark} \hlt{Conventional Central Bank Tools}
\begin{enumerate}[label=\roman*.]
\setlength{\itemsep}{0pt}
\item \textbf{Policy Rate}: Set by changing overnight rate at which banks lend and borrow to meet reserve requirements. The primary tool in normal times.
\item \textbf{Reserve Ratio}: Raising reserve ratio decreases money multiplier and contracts money supply.
\item \textbf{Open Market Operations (OMO)}: Buying and selling government bonds in local currency. Repos are short-term agreements where the central bank agrees to repurchase securities within a specified time.
\item \textbf{Currency Intervention}:
\begin{enumerate}[label=\arabic*.]
\setlength{\itemsep}{0pt}
\item Sterilised intervention: Trading own currency via FX reserves without changing monetary base
\item Unsterilised intervention: Allows for changes in monetary base
\end{enumerate}
\end{enumerate}
\end{remark}

\begin{remark} \hlt{Taylor's Rule}\\
Maps macroeconomic deviations to policy stance:
\begin{equation*}
i_t = r^* + \pi_t + \alpha (\pi_t - \pi^*) + \beta (y_t - y_t^*)
\end{equation*}
where $r^*$ is the neutral real rate (unobservable, typically estimated 2--3\%), $\pi_t$ is current inflation, $\pi^*$ is the inflation target, $(y_t - y_t^*)$ is the output gap, and $\alpha, \beta$ are response coefficients (typically 0.5).\\
The rule states that the nominal policy rate must rise more than one-to-one with inflation (Taylor Principle); otherwise real rates fall, inflation becomes self-reinforcing, and expectations de-anchor.\\
Central banks treat Taylor's rule as a benchmark, communication tool, and political shield.
\end{remark}

\begin{remark} \hlt{The Impossible Trinity}\\
A country cannot simultaneously maintain:
\begin{enumerate}[label=\roman*.]
\setlength{\itemsep}{0pt}
\item Fixed exchange rate
\item Free capital flows
\item Independent monetary policy
\end{enumerate}
This constraint explains currency crises (ERM 1992, Asian crisis 1997), forced rate hikes or cuts, FX pegs breaking, and why EM central banks behave "irrationally" under stress. Countries must choose two of three.
\end{remark}

\begin{remark} \hlt{Central Bank Balance Sheet}
\begin{enumerate}[label=\roman*.]
\setlength{\itemsep}{0pt}
\item \textbf{Monetary Base}: Sum of currency in circulation and reserves held at the central bank. Can only be controlled via open market operations.
\item \textbf{Money Supply}: Changes in money supply are leading predictors of GDP growth and inflation. The Quantity Theory of Money:
\begin{equation*}
M \times V = P \times Y
\end{equation*}
where $M$ is money supply, $V$ is velocity of money (turnover rate), $P$ is price level, $Y$ is real output.
\end{enumerate}
Post-2008, central bank balance sheets expanded dramatically. The Fed releases weekly H.4.1 report showing sources and uses of reserves.
\end{remark}

\begin{remark} \hlt{Types of Money Supply}
\begin{enumerate}[label=\roman*.]
\setlength{\itemsep}{0pt}
\item $M0$: Physical currency in circulation (notes and coins)
\item $M1$: $M0$ + demand deposits, checking accounts, traveler's checks
\item $M2$: $M1$ + savings deposits, money market accounts, small time deposits, retail money market funds
\item $M3$: $M2$ + large time deposits, institutional money market funds, repos, eurodollars
\end{enumerate}
Central banks primarily control $M0$; broader measures depend on banking system credit creation.
\end{remark}

\begin{remark} \hlt{Zero Lower Bound and the Liquidity Trap}\\
When central bank rate is at or near zero:
\begin{enumerate}[label=\roman*.]
\setlength{\itemsep}{0pt}
\item Rate cuts lose marginal effectiveness
\item Expectations dominate outcomes
\item Forward guidance becomes critical
\item Fiscal policy starts to matter more
\end{enumerate}
Yield curves flatten naturally; carry trades distort; volatility is suppressed artificially.\\
Some central banks have gone to negative rates, testing the effective lower bound.
\end{remark}

\begin{remark} \hlt{Quantitative Easing (QE)}\\
Central banks purchase financial assets (government bonds, MBS, corporate bonds), creating electronic reserves to expand the balance sheet. QE works through multiple channels:
\begin{enumerate}[label=\roman*.]
\setlength{\itemsep}{0pt}
\item Portfolio rebalancing (investors pushed into riskier assets)
\item Risk premia compression (term premium, credit spreads)
\item Signalling effects (commitment to low rates)
\item Exchange rate channel (currency weakening)
\end{enumerate}
Bank lending does not mechanically follow reserve creation. Transmission depends on credit demand and bank capital. Effectiveness diminishes with successive rounds.\\
Fed research shows QE1 lowered $10$-year yields by $100$bp; QE2 by only $14$bp.
\end{remark}

\begin{remark} \hlt{Hawks and Doves}\\
Central bankers have individual biases that shape their policy views:
\begin{enumerate}[label=\roman*.]
\setlength{\itemsep}{0pt}
\item \textbf{Hawks}: Wary of inflation, favour tighter policy, defend currency value
\item \textbf{Doves}: Favour growth and employment, tolerate higher inflation, support looser policy
\end{enumerate}
When a known hawk makes dovish statements (or vice versa), it signals a significant shift and moves markets more than statements consistent with known bias.\\
Tracking individual central banker positions is essential for anticipating policy shifts.
\end{remark}

\begin{remark} \hlt{Central Bank Communication}\\
Central bank communication is a policy tool in itself:
\begin{enumerate}[label=\roman*.]
\setlength{\itemsep}{0pt}
\item \textbf{Statements}: Released after each meeting; market moves on hawkish/dovish tone changes
\item \textbf{Minutes}: Released with lag ($2$ to $3$ weeks); reveal vote splits and internal debates
\item \textbf{Press Conferences}: Chair/President Q\&A; tone and body language matter
\item \textbf{Projections}: Dot plots (Fed), staff forecasts; reveal expected rate path
\item \textbf{Speeches}: Individual members signalling; watch for coordinated messaging
\end{enumerate}
Example: BOE June 2012 minutes revealed $5$--$4$ vote against QE (vs $8$--$1$ in May), correctly predicting QE expansion in July.
\end{remark}

\begin{remark} \hlt{Major Central Banks}
\begin{enumerate}[label=\roman*.]
\setlength{\itemsep}{0pt}
\item \textbf{Federal Reserve (Fed)}: Dual mandate (price stability, full employment). FOMC meets $8$ times/year with $12$ voting members ($7$ governors, NY Fed, $4$ rotating).\\
Most important central bank globally due to USD reserve status.
\item \textbf{European Central Bank (ECB)}: Price stability mandate (HICP $<2\%$). Governing Council of $25$ members. Key rates: Main Refinancing Operations (MRO), deposit facility rate. Tools include LTRO, TLTRO, OMT, APP.
\item \textbf{Bank of Japan (BOJ)}: Targets $2\%$ inflation (chronically undershoots). Policy Board of $9$ members. Pioneer of QE, yield curve control (YCC), negative rates.
\item \textbf{Bank of England (BOE)}: $2\%$ inflation target. MPC of $9$ members with individual accountability. First major central bank to publish inflation forecasts.
\item \textbf{Swiss National Bank (SNB)}: Inflation target $<2\%$. Notable for 2011--2015 EUR/CHF floor at $1.20$ and subsequent abandonment. Massive FX reserve accumulation.
\item \textbf{People's Bank of China (PBOC)}: Multiple objectives including growth, stability, FX management. Manages CNY within trading band. Uses reserve requirements actively.
\end{enumerate}
\end{remark}

\begin{remark} \hlt{ECB-Specific Tools}
\begin{enumerate}[label=\roman*.]
\setlength{\itemsep}{0pt}
\item \textbf{LTRO/TLTRO}: Long-Term Refinancing Operations provide multi-year funding to banks against collateral. $3$-year LTROs in $2011$ to $2012$ injected over EUR$1$ trillion.
\item \textbf{OMT (Outright Monetary Transactions)}: Announced September $2012$. Unlimited bond purchases for countries in EFSF/ESM programs. Never actually used but stabilised spreads.
\item \textbf{APP (Asset Purchase Programme)}: QE-equivalent. Purchases of government bonds, covered bonds, corporate bonds, ABS.
\item \textbf{TARGET2}: Payment system showing intra-Eurozone imbalances. German TARGET2 claims rose to EUR$700$B+ during crisis, reflecting capital flight from periphery.
\end{enumerate}
\end{remark}

\begin{remark} \hlt{Fed Crisis-Era Programmes}\\
During and after 2008, the Fed deployed unprecedented tools:
\begin{enumerate}[label=\roman*.]
\setlength{\itemsep}{0pt}
\item \textbf{Central Bank Swap Lines}: USD liquidity to foreign central banks to prevent LIBOR spikes
\item \textbf{TAF (Term Auction Facility)}: $28$--$84$ day unsecured lending without discount window stigma
\item \textbf{TSLF (Term Securities Lending Facility)}: Treasury lending against lower-quality collateral
\item \textbf{CPFF (Commercial Paper Funding Facility)}: Commercial paper purchases to prevent market freeze
\item \textbf{TALF (Term ABS Loan Facility)}: Lending against consumer ABS to support credit markets
\item \textbf{Agency MBS Purchases}: $\$1.25$ trillion in QE1 alone
\end{enumerate}
Understanding these tools helps anticipate future crisis responses.
\end{remark}

\begin{remark} \hlt{Case Study: Black Wednesday (1992)}\\
Classic example of the Impossible Trinity and central bank limitations:
\begin{enumerate}[label=\roman*.]
\setlength{\itemsep}{0pt}
\item UK joined ERM in October $1990$, pegging GBP to DEM at $1/2.95$ DM
\item UK need lower rates (weak growth, falling inflation); Germany need higher (reunification, rising inflation)
\item Bundesbank raised rates in July $1992$, forcing BOE to defend peg via intervention
\item George Soros correctly predicted BOE had limited reserves and could not sustain sterilised intervention
\item September $16$, $1992$: BOE raised rates to $15\%$, then abandoned ERM same day
\item GBP lost $10\%$ vs DEM in less than a week; Soros made $\$1B+$ profit
\end{enumerate}
Lesson: Central banks cannot fight fundamentals indefinitely. FX pegs are options that can be exercised against the central bank.
\end{remark}

\subsubsection{Economic Data Releases and Demographics}

\begin{remark} \hlt{Data Releases and Market Impact}\\
Data releases are central bank inputs, not just statistics. Markets price expectations before the release; the move comes from the surprise, not the level. Key principles:
\begin{enumerate}[label=\roman*.]
\setlength{\itemsep}{0pt}
\item \textbf{Surprise Direction}: Above/below consensus drives immediate price action. Magnitude of surprise matters more than absolute level.
\item \textbf{Revision Risk}: Prior releases are frequently revised. Initial release moves markets; revisions move markets again if material.
\item \textbf{Hierarchy}: Not all data is equal. NFP and CPI dominate in the US; PMIs dominate globally due to frequency and forward-looking nature.
\item \textbf{Regime Dependence}: What data matters depends on the regime. In low-inflation regimes, growth dominates. In high-inflation regimes, inflation dominates everything.
\end{enumerate}
Back-testing asset responses to past data releases (queries) can inform trade sizing and direction. However, false positives are common. If wrong, cut quickly.
\end{remark}

\begin{remark} \hlt{Growth Indicators}\\
Growth data sets the directional bias for risk assets, but markets trade changes, not levels.
\begin{enumerate}[label=\roman*.]
\setlength{\itemsep}{0pt}
\item \textbf{GDP (Headline, QoQ, YoY)}: Measures total production in current or constant prices (nominal, real GDP). Released quarterly with preliminary, revised, and final prints. Backward-looking but sets narrative.
\item \textbf{Purchasing Managers' Indices (PMIs)}: Survey business managers on employment, new orders, inventories, production, supplier deliveries, prices, backlogs, imports, and exports on scale of better/same/worse vs prior month. Above 50 = expansion; below 50 = contraction. Most important leading indicator.
\item \textbf{Industrial Production}: Measures growth in manufacturing, utilities, and mining. More cyclical than services. Released monthly. Leading indicator for GDP forecasting.
\item \textbf{Trade Volumes, Baltic Dry Index (BDI)}: Shipping costs for dry bulk cargo. Leading indicator on global trade activity since it cannot be speculated (no futures market).
\end{enumerate}
PMIs and surveys move markets more than GDP because they are higher frequency and forward-looking.\\
Growth surprises affect equities (earnings expectations), commodities (demand), FX (via growth differentials).
\end{remark}

\begin{remark} \hlt{Inflation}\\
Inflation data is primarily a policy variable. It constrains or enables central bank action.
\begin{enumerate}[label=\roman*.]
\setlength{\itemsep}{0pt}
\item \textbf{Consumer Price Index (CPI)/Core CPI}: Prices paid by consumers for basket of goods and services. Core excludes food and energy (volatile). Central banks target core.
\item \textbf{Producer Price Index (PPI)}: Average change in selling prices received by domestic producers. Leads CPI by $1$--$3$ months as input costs pass through.
\item \textbf{PCE (Personal Consumption Expenditures)}: Fed's preferred inflation measure. Differs from CPI in weights and methodology.
\item \textbf{Inflation Expectations}: Market-based (breakevens from TIPS) or survey-based (Michigan, NY Fed). Central banks watch expectations closely. De-anchoring is a regime change.
\end{enumerate}
Inflation surprises directly affect real yields, curve shape, FX via real rate differentials.\\
Inflation matters most when central banks are near constraints (ZLB, credibility questioned).
\end{remark}

\begin{remark} \hlt{Employment Data}\\
Labor data bridges growth, inflation, and central bank reaction function.
\begin{enumerate}[label=\roman*.]
\setlength{\itemsep}{0pt}
\item \textbf{Non-Farm Payrolls (NFP)}: Number of workers in US excluding farms, private households, nonprofits, self-employed, and military. Released first Friday of each month. Among highest-volatility releases for rates, FX, equities.
\item \textbf{Unemployment Rate}: Unemployed as percentage of labor force. Headline number; can be distorted by participation changes.
\item \textbf{Participation Rate}: Labor force as percentage of civilian population. Structural decline (aging) vs cyclical decline (discouraged workers) matters.
\item \textbf{Wage Growth (Average Hourly Earnings)}: More important than job count for inflation outlook. 
\item \textbf{JOLTS (Job Openings and Labor Turnover Survey)}: Shows job openings, hires, quits. Quits rate indicates labor market tightness and worker confidence.
\end{enumerate}
Wage growth drives services inflation, which is stickier than goods inflation.\\
Fed's dual mandate makes employment data directly policy-relevant.
\end{remark}

\begin{remark} \hlt{External Accounts}\\
External balances define structural currency pressure and financing needs.
\begin{enumerate}[label=\roman*.]
\setlength{\itemsep}{0pt}
\item \textbf{Current Account}: Net exports + net foreign income + transfer payments. Deficit countries need capital inflows or currency depreciation; surplus accumulate reserves and have structurally stronger currencies.
\item \textbf{Capital Account/Financial Account}: Portfolio flows, FDI, banking flows. Can finance current account deficits but creates vulnerabilities (sudden stop risk).
\item \textbf{Trade Balance}: Exports minus imports. Incremental changes in exports or imports can have big impact on trade balance as percentage of GDP.
\item \textbf{Terms of Trade (ToT)}: Export prices divided by import prices. Improves when export prices rise relative to imports. Currency depreciation typically worsens ToT.
\item \textbf{FX Reserves}: Central bank holdings of foreign currency and gold. Countries with large reserves can intervene; watching reserve changes indicates intervention.
\end{enumerate}
Persistent current-account deficits require higher yields to attract capital or eventual currency adjustment.
\end{remark}

\begin{remark} \hlt{Government/Fiscal Indicators}\\
Fiscal data determines long-term sustainability, not short-term price action.
\begin{enumerate}[label=\roman*.]
\setlength{\itemsep}{0pt}
\item \textbf{Budget Balance}: Revenues minus expenditures. Deficit = borrowing required. Deficit/GDP ratio matters more than absolute level.
\item \textbf{Debt-to-GDP}: Stock measure of accumulated borrowing. Matters more for EM and Euro periphery than for reserve currency issuers.
\item \textbf{Primary Balance}: Budget balance excluding interest payments. Shows underlying fiscal stance; positive primary balance means debt/GDP can stabilise if growth exceeds interest rate.
\item \textbf{Debt Sustainability}: Interest payments as \% of revenue. Rising share means deteriorating sustainability.
\end{enumerate}
Fiscal deterioration matters when markets doubt financing ability or central bank independence.\\
Sovereign risk appears first in bond spreads, CDS, FX (especially EM).
\end{remark}

\begin{remark} \hlt{Consumption Indicators}\\
Consumption drives developed-market growth ($70\%$+ of US GDP).
\begin{enumerate}[label=\roman*.]
\setlength{\itemsep}{0pt}
\item \textbf{Retail Sales}: Consumer demand for finished goods. Durable vs non-durable breakdown matters.
\item \textbf{Consumer Confidence}: Survey of expected financial situation. Conference Board and Michigan surveys in US. Leading indicator, confidence leads spending, but noisy.
\item \textbf{Personal Income and Spending}: Income growth enables spending growth. Savings rate = income minus spending.
\item \textbf{Savings Rate}: Higher savings = lower current consumption but more sustainable. US savings rate has declined structurally; Asian savings rates remain high.
\end{enumerate}
Consumption data is most relevant for equity indices, cyclical sectors, domestic currencies.\\
US consumer accounts for $\sim 15\%$ of global GDP directly through consumption.
\end{remark}

\begin{remark} \hlt{Housing Indicators}\\
Housing is cyclically important and credit-sensitive.
\begin{enumerate}[label=\roman*.]
\setlength{\itemsep}{0pt}
\item \textbf{Building Permits}: Lead housing starts since permits are required before construction. Most forward-looking housing indicator.
\item \textbf{Housing Starts}: Counted when construction begins. Gauge of housing demand and construction activity.
\item \textbf{Existing/New Home Sales}: Volume of transactions. Prices vs volumes can diverge.
\item \textbf{Home Price Indices}: Case-Shiller, FHFA. Lag actual market conditions.
\item \textbf{Household Formations}: Structural driver of housing demand. Post-crisis decline reduced demand.
\end{enumerate}
Housing matters for rate sensitivity, consumer wealth effects, and financial stability.
\end{remark}

\begin{remark} \hlt{Business and Industry Indicators}\\
These indicators show where growth is coming from and where it's heading.
\begin{enumerate}[label=\roman*.]
\setlength{\itemsep}{0pt}
\item \textbf{Manufacturing vs Services PMI}: Manufacturing is more cyclical and globally linked. Services dominate GDP but move slower.
\item \textbf{New Orders}: Most forward-looking PMI subcomponent. Rising new orders = future production.
\item \textbf{Inventories}: High inventories relative to sales = future production cuts. Inventory cycles matter for commodities and industrial equities.
\item \textbf{Capacity Utilisation}: Actual output relative to potential. Over $100\%$ = overheating, inflation risk. Well under $100\%$ = slack, deflation risk.
\item \textbf{Durable Goods Orders}: Capital expenditure intentions. Core capital goods orders (ex-aircraft, ex-defence) is key business investment indicator.
\end{enumerate}
ISM New Orders minus Inventories spread is a reliable leading indicator of manufacturing cycles.
\end{remark}

\begin{remark} \hlt{Demographics}\\
Slowest-moving macro variable, but the hardest constraint. Demographics cannot be QE'd.
\begin{enumerate}[label=\roman*.]
\setlength{\itemsep}{0pt}
\item \textbf{Working-Age Population ($15$--$64$)}: Drives potential growth. Many developed countries have peaked or will peak soon.
\item \textbf{Prime Saving Age ($35$--$69$)}: Peak saving years. Countries past this peak see declining savings and current accounts.
\item \textbf{Dependency Ratio}: Non-working population (young + old) divided by working population. Higher ratio = more fiscal pressure, lower savings.
\item \textbf{Median Age}: Summary statistic for demographic structure. Europe/Japan: 40s; US: 38; EM: $20$s--$30$s.
\item \textbf{Retirement Age vs Life Expectancy}: Gap determines pension system sustainability. Life expectancy rising; retirement ages barely moving.
\end{enumerate}
Demographics affect long-term growth, savings rates, current accounts, equilibrium real rates.
\end{remark}

\begin{remark} \hlt{Demographic Implications by Country}\\
Peak working-age populations determine structural trajectories:
\begin{enumerate}[label=\roman*.]
\setlength{\itemsep}{0pt}
\item \textbf{Already peaked}: Germany ($2006$), Japan ($2016$), Netherlands ($1989$), UK ($2009$), US ($2007$)
\item \textbf{Peaking soon}: China ($2032$), Korea ($2023$), Russia ($2025$), Spain ($2022$)
\item \textbf{Long runway}: India ($2050$), Indonesia ($2050$), Nigeria ($2050$), Philippines ($2050$)
\end{enumerate}
Countries past peak will see declining current accounts, rising dependency ratios, pressure on fiscal positions.\\
Jean-Claude Trichet: "Current account balance is an important summary indicator that signals losses of competitiveness and emerging imbalances."
\end{remark}

\begin{remark} \hlt{Building Queries for Data Events}\\
Queries back-test asset behaviour around past occurrences of specific events:
\begin{enumerate}[label=\roman*.]
\setlength{\itemsep}{0pt}
\item Identify the event (e.g., BOE QE announcement, Fed rate hike, NFP beat)
\item Find historical dates with similar conditions
\item Calculate asset returns +$1$ day, +$2$ days, +$1$ week, +$1$ month after event
\item Compute hit rates (percentage of times asset moved in expected direction)
\item Assess risk/reward: average return vs standard deviation
\end{enumerate}
Queries are powerful for short-term event trading and understanding portfolio exposure to upcoming releases.\\
Caveat: False positives are common. If position moves against you immediately after event, cut quickly.
\end{remark}

\begin{remark} \hlt{Key Data Release Calendar (US)}\\
Monthly cadence of major releases:
\begin{enumerate}[label=\roman*.]
\setlength{\itemsep}{0pt}
\item \textbf{First Friday}: Employment Report (NFP, unemployment rate, wages)
\item \textbf{Mid-month}: CPI, Retail Sales, Industrial Production
\item \textbf{Third week}: Housing Starts, Building Permits
\item \textbf{Late month}: GDP (quarterly), PCE, Durable Goods
\item \textbf{First business day}: ISM Manufacturing PMI
\item \textbf{Third business day}: ISM Services PMI
\end{enumerate}
FOMC meetings (8 per year) and ECB meetings (6 per year) often dominate all other releases.\\
Economic calendars (Bloomberg ECO, Reuters, Trading Economics) are essential tools.
\end{remark}

