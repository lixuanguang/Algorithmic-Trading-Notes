\subsection{Transmission and Cross-Asset Mapping}

\subsubsection{Building Blocks of Global Macro}

\begin{remark} \hlt{Macro Strategy Product Groups}\\
The four core product groups in global macro are:
\begin{enumerate}[label=\roman*.]
\setlength{\itemsep}{0pt}
\item \textbf{Fixed Income} (FI) (Policy and Growth Sensor): encodes growth expectations, inflation expectations, central bank credibility. Yield curves summarise the entire macro narrative; steepening means growth/reflation; flattening or inversion means slowdown/recession. Fixed income tells what the market thinks will happen before it happens.
\item \textbf{Foreign Exchange} (FX) (Adjustment Mechanism): FX reflects relative, not absolute, macro conditions. Sensitive to rate differentials, capital flows, balance of payments, risk-on/risk-off regimes. FX absorbs and redistributes macro shocks across countries.
\item \textbf{Equities} (Growth Expression): equities are long growth, long liquidity, short uncertainty. Equity indices aggregate earnings expectations, discount rates, risk appetite. Volatility (VIX) is a second-order macro signal, not just an equity signal. Equities express the confidence (or fear) embedded in the macro outlook.
\item \textbf{Commodities} (Inflation and Real-Economy Link): driven by physical supply and demand, inventories, geopolitics. Highly sensitive to growth cycles, inflation shocks, currency moves (especially USD). Commodities anchor macro views in real economic constraints.
\end{enumerate}
These are the most liquid markets globally, which allows macro strategies to scale to large AUM while maintaining execution quality.
\end{remark}

\begin{remark} \hlt{Relationship Between Assets}\\
These relationships are regime-dependent, not permanent laws:
\begin{enumerate}[label=\roman*.]
\setlength{\itemsep}{0pt}
\item \textbf{Growth Shocks}: rates rise, equities rally, cyclicals outperform, commodities rise. FX favours growth-linked currencies (AUD, CAD, EM FX).
\item \textbf{Recession Shocks}: rates fall, curves flatten/invert, equities sell off, USD strengthens (flight to safety), commodities weaken (except gold).
\item \textbf{Inflation Shocks}: nominal rates up, real rates determine equity impact. Commodities outperform. FX punishes low-credibility central banks.
\item \textbf{Liquidity Shocks}: USD strengthens sharply, credit spreads widen, correlations spike toward 1, volatility explodes. Cross-currency basis blows out.
\end{enumerate}
A single macro thesis should have expressions across multiple asset classes, each with different risk profiles and convexity characteristics.
\end{remark}

\begin{remark} \hlt{Asset Correlation Dynamics}\\
Asset correlations change under stress. Diversification fails when needed most, where assets that appear uncorrelated in calm periods may collapse together in crises.\\
Macro traders profit when they anticipate correlation breakdowns and position for nonlinear regime shifts. This is why macro portfolios use convex instruments (options), size conservatively, and focus on liquidity.\\
Traders must know what assets they are trading in aggregate. If entire portfolio has assets with high correlation to one another, they can be stopped out of all positions more easily if the market turns against them.
\end{remark}

\begin{remark} \hlt{Position Sizing Framework}\\
Position sizing is arguably more important than knowing what to buy or sell. The first rule of investing is to preserve capital and avoid loss.\\
Position sizing aims to adjust each position for volatility. This normalizes all positions such that losses are more predictable and have the same probability of occurring:
\begin{enumerate}[label=\roman*.]
\setlength{\itemsep}{0pt}
\item Calculate $N$ = $20$-day exponential moving average of True Range
\item True Range = max(High $-$ Low, High $-$ Previous Close, Previous Close $-$ Low)
\item Use $2N$ for stop-loss calculation (two standard deviations) to avoid being stopped out by normal volatility
\item With $2N$, there is approximately 2\% chance of being stopped out on any given day (vs.\ $16\%$ with $1N$)
\end{enumerate}
This methodology, developed by the Turtle Traders, allows traders to take positions in a broad range of markets while monitoring risk consistently.
\end{remark}

\begin{remark} \hlt{Risk/Reward Discipline}\\
Given human error and transaction costs, traders can only hope to be right about 50\% of the time. Having good risk/reward on each trade will make or break long-term success.\\
As a baseline, always strive for a 1:3 risk-to-reward ratio or higher:
\begin{enumerate}[label=\roman*.]
\setlength{\itemsep}{0pt}
\item For every trade risking 25 basis points, expect to make at least 75 basis points
\item Set targets before the trade is executed to avoid biases of irrational human decision making
\item Log reasons for each trade in a trade journal
\item If original thesis no longer holds true, take trade off, but do not create excuses for taking trades off early
\end{enumerate}
\end{remark}

\begin{remark} \hlt{Gap Risk}\\
Gap risk is one of the biggest risks when sizing positions. Gap risk occurs when prices change without any trading in between (e.g., overnight, over weekends). This risk is especially acute in:
\begin{enumerate}[label=\roman*.]
\setlength{\itemsep}{0pt}
\item EM currencies and assets (less liquid, more event-driven)
\item Positions held over weekends or holidays
\item Periods of elevated geopolitical risk
\item Earnings/data release windows
\end{enumerate}
\end{remark}

\begin{remark} \hlt{Risk Utilization}\\
Risk-taking is nonlinear. Even with $50\%$ conviction, should not run $50\%$ of risk utilisation.
\begin{enumerate}[label=\roman*.]
\setlength{\itemsep}{0pt}
\item Base case: set a maximum VaR limit (e.g., $5\%$ daily VaR at $95\%$ confidence)
\item If up $5\% $ on the year, allow VaR limit to increase to $6\%$ 
\item If down $2.5\%$ on the year, decrease VaR limit to $4\%$
\item At the start of each new year, reset to $0\%$ return regardless of prior year performance
\end{enumerate}
The intention is to create steady returns while preserving capital and avoiding blow-up risk.
\end{remark}

\begin{remark} \hlt{Stress Testing}\\
Stress testing aims to go beyond VaR to find potential worst-case scenarios:
\begin{enumerate}[label=\roman*.]
\setlength{\itemsep}{0pt}
\item Run historical back-test on P\&L for previous $500$ trading days ($\approx 2$ years)
\item Sort from best to worst days; examine the average of the $5$ worst days (worst $1\%$)
\item In periods of low volatility, also test against extreme historical events: October $1987$ crash, $1998$ LTCM/Russian default, $2008$ financial crisis, $2011$ Japan earthquake, $2015$ CNY devaluation, $2020$ COVID crash
\end{enumerate}
Knowing maximum loss outside of VaR helps traders understand exposure to tail events.
\end{remark}

\begin{remark} \hlt{Performance Metrics}\\
Key metrics for evaluating macro trading performance:
\begin{enumerate}[label=\roman*.]
\setlength{\itemsep}{0pt}
\item \textbf{Sharpe Ratio}: $\frac{E[R] - R_f}{\sigma}$ --- risk-adjusted return; higher is better. Assumes normal distribution.
\item \textbf{Sortino Ratio}: Sharpe with only downside volatility, differentiating between good and bad volatility.
\item \textbf{Maximum Drawdown}: peak-to-trough decline; critical for understanding worst-case capital loss.
\item \textbf{Value at Risk (VaR)}: $95\%$ or $99\%$ daily VaR (expected loss exceeded on $1$ in $20$ or $100$ days).
\end{enumerate}
When evaluating strategies, test Sharpe ratios rather than raw returns. Break down P\&L by asset class, duration, and conviction level to identify strengths and weaknesses.
\end{remark}

\begin{remark} \hlt{Macro View Thesis}\\
A macro view expressed in only one asset class is incomplete. A single macro thesis should have multiple expressions, each with different risk profiles:
\begin{enumerate}[label=\roman*.]
\setlength{\itemsep}{0pt}
\item \textbf{Growth view}: rates (duration), equities (beta), FX (pro-cyclical currencies), commodities (cyclicals)
\item \textbf{Inflation view}: commodities (energy, agriculture), breakevens, FX (commodity currencies), TIPS
\item \textbf{Crisis/Deflation view}: bonds (duration), USD/JPY/CHF (havens), volatility (VIX, swaptions), gold
\item \textbf{Policy divergence view}: FX (rate differentials), yield curve spreads, relative equity indices
\end{enumerate}
Having multiple expressions allows for confirmation across asset classes and provides natural hedging if one leg underperforms.
\end{remark}

\subsubsection{Foreign Exchange in Global Macro}

FX plays three special roles simultaneously:
\begin{enumerate}[label=\roman*.]
\setlength{\itemsep}{0pt}
\item Macro shock absorber 
\item Macro transmission channel 
\item Macro expression instrument
\end{enumerate}

\begin{remark} \hlt{FX Market Structure}\\
The FX market is the world's largest and most liquid market, and operates 24 hours per day, 5.5 days per week across global financial centers (London, New York, Tokyo, Singapore, Hong Kong).\\
Key participants include: central banks (intervention, reserve management), commercial banks (market-making, proprietary trading), corporations (hedging trade flows), asset managers (currency overlay, alpha generation), hedge funds (macro speculation, carry trades), and retail traders.\\
The market is predominantly OTC, with spot, forwards, and swaps as the main instruments. FX futures trade on exchanges (CME) but represent a small fraction of total volume.
\end{remark}

\begin{remark} \hlt{Role of the US Dollar}\\
Global macro is de facto USD-centric. USD is the global reserve currency, most commodities are USD-priced, global funding markets clear in USD, and cross-border leverage is USD-denominated.\\
The USD is involved in approximately 85\% of all FX transactions globally. Many ``local'' macro trades are actually USD liquidity trades in disguise.\\
Stress often appears first in US Dollar Index (DXY), cross-currency basis, EM FX before local assets.\\
The DXY Index measures USD against a basket of six major currencies: EUR (57.6\%), JPY (13.6\%), GBP (11.9\%), CAD (9.1\%), SEK (4.2\%), CHF (3.6\%). The DXY is not trade-weighted and over-represents EUR.
\end{remark}

\begin{remark} \hlt{Special Drawing Rights (SDRs)}\\
SDRs are international reserve assets created by the IMF to supplement member countries' official reserves. The SDR basket is reviewed every five years. SDRs serve as a unit of account for IMF transactions and provide insight into official reserve diversification trends.
\end{remark}

\begin{remark} \hlt{Currency Regimes}\\
Currency regimes matter more than models. FX does not behave uniformly. Regime dominates signal.
\begin{enumerate}[label=\roman*.]
\setlength{\itemsep}{0pt}
\item \textbf{Free-floating}: price discovery happens in FX
\item \textbf{Managed/pegged}: pressure builds elsewhere such as in rates, reserves, capital controls
\item \textbf{Hard pegs/Currency boards}: FX looks stable until it breaks violently
\end{enumerate}
FX valuation only works within regime constraints. Ignore the regime and ``cheap'' can stay cheap forever.
\end{remark}

\begin{remark} \hlt{Central Bank Intervention}\\
Central banks intervene in FX markets for multiple objectives:
\begin{enumerate}[label=\roman*.]
\setlength{\itemsep}{0pt}
\item Defend a peg or managed float band
\item Smooth excessive volatility
\item Counter speculative attacks
\item Accumulate or deploy reserves
\end{enumerate}
Intervention effectiveness depends on: reserve adequacy, policy credibility, coordination with monetary policy, and market liquidity conditions. Intervention against fundamental pressures typically fails. Watch for reserve depletion as a leading indicator of forced adjustment.
\end{remark}

\begin{remark} \hlt{Valuation Techniques}
\begin{enumerate}[label=\roman*.]
\setlength{\itemsep}{0pt}
\item Market Implied and Risk-Appetite Signals
\begin{enumerate}[label=\arabic*.]
\setlength{\itemsep}{0pt}
\item Equity Index Price Performance: strong local equities (vs peers) attract inflows and support currency; sharp equity underperformance foreshadows FX weakness via risk-off deleveraging and foreign selling.
\item Credit/Sovereign Risk: widening sovereign spreads/CDS signals default/policy credibility/funding stress, weakening FX or raising devaluation risk premia. Also a regime detector for EM FX.
\item Sentiment/Positioning: call-put skew on three-month risk reversals. Acts as a multiplier. Can create overshoots/undershoots around fair value in crowded carry or crisis episodes.
\end{enumerate}
\item Trade and Competitiveness Fundamentals
\begin{enumerate}[label=\arabic*.]
\setlength{\itemsep}{0pt}
\item Trade Balance/External Account: structurally strong trade position often supports FX; persistent deficits can make the currency reliant on financing.
\item Trade-Weighted Index (TWI): reflects currency's effective competitiveness against trading partners---the currency basket that matters for the real economy.
\item Economic Activity: strong activity can support FX via higher expected returns and inflows, but it can also worsen the trade balance.
\item Export Partner Growth: mapping who buys exports and whether those buyers are expanding/contracting. Export partner growth leads exporter's FX through expected trade receipts.
\end{enumerate}
\item Macro Sustainability Indicators
\begin{enumerate}[label=\arabic*.]
\setlength{\itemsep}{0pt}
\item Debt-to-GDP: high public debt with FX liabilities or weak domestic savings can increase risk premia and currency vulnerability, as it tightens the policy constraint set (inflation vs austerity vs default).
\item Current Account Balance: persistent deficits require capital inflows; sudden stops trigger sharp FX adjustment.
\item FX Reserves: adequacy (months of import cover, short-term debt coverage) determines policy space.
\end{enumerate}
\item Long-Term Valuation Anchors
\begin{enumerate}[label=\arabic*.]
\setlength{\itemsep}{0pt}
\item Purchasing Power Parity (PPP): the idea that exchange rates should adjust so that identical goods cost the same in different countries. In practice, PPP convergence can take years or decades.
\item GDP Per Capita: structural development/productivity proxy. Richer, higher productivity economies tend to sustain stronger real exchange rates over time (Balassa-Samuelson effect).
\end{enumerate}
\item Carry Indicators
\begin{enumerate}[label=\arabic*.]
\setlength{\itemsep}{0pt}
\item OIS Differential: compares respective policy rates of two countries. Carry is the interest rate differential between two currencies.
\item LIBOR/SOFR Differential: reflects unsecured interbank rates. Can be decomposed into risk-free rate + credit spread component. Using relative spreads gives a sense of direction; interest rate differential drives demand for a currency.
\item Risk-Adjusted Carry: adjusts carry for volatility. Outperforms standard carry trades and is useful in cross-currency analysis:
\begin{equation*}
\text{Carry-to-Risk Ratio} = \frac{\text{3M Carry (Rate Differential)}}{\text{30D Realized Volatility}}
\end{equation*}
Within high-yielding currency pairs, one would ideally want the highest carry-to-risk ratio. Volatility and carry components are always in flux, so adjustments should be made frequently.
\end{enumerate}
\end{enumerate}
\end{remark}

\begin{remark} \hlt{Carry Trade Mechanics}\\
Carry strategies are typically executed by borrowing in lower-yielding currencies and buying higher-yielding currencies. The carry trade is a crowded trade; in large risk-off markets, the strategy suffers significant losses as positions unwind simultaneously.\\
Carry is the return earned on holding a currency, assuming the exchange rate is held constant. In practice, exchange rate moves often dominate carry returns. Historically, carry strategies exhibit positive returns on average but with significant negative skewness (small gains, occasional large losses).
\end{remark}

\begin{remark} \hlt{Carry Trade Risk Factors}\\
Key risks to carry trades include:
\begin{enumerate}[label=\roman*.]
\setlength{\itemsep}{0pt}
\item Crowding Risk: carry trades are popular, leading to violent unwinds during risk-off episodes.
\item Gap Risk: EM currencies can gap significantly on weekends or during crises, bypassing stop-losses.
\item Regime Shifts: central bank policy changes can rapidly alter rate differentials.
\item Correlation Breakdown: in stress, all carry currencies tend to sell off together against funding currencies.
\item Liquidity Risk: EM FX can become illiquid precisely when exits are needed most.
\end{enumerate}
Finding the carry-to-risk ratio is useful, as is comparing all currency pairs on a relative basis. Measuring the standard error of the sample and finding the most attractive pairs helps in mean-reverting carry strategies.
\end{remark}


\subsubsection{Equities in Global Macro}

In macro, a trader should understand equity in terms of market behaviour in the country being traded, the sectors that make up those indices, and correlation risk.

\begin{remark} \hlt{Top-Down Macro Approach to Equities}\\
Many macro traders take a top-down approach to trading equities:
\begin{enumerate}[label=\roman*.]
\setlength{\itemsep}{0pt}
\item Start with a global view
\item Narrow to countries expected to outperform
\item Select the best sectors within those countries
\item Optionally, select best companies within sectors
\end{enumerate}
If the investor has a knack for knowing where the next opportunity lies, the most reward typically comes from concentrated exposure rather than diversified index holdings.
\end{remark}

\begin{remark} \hlt{Major Equity Indices by Region}
\begin{enumerate}[label=\roman*.]
\setlength{\itemsep}{0pt}
\item United States: S\&P 500, Nasdaq, DJIA, Russell 2000, Wilshire 5000
\item Europe: Euro Stoxx 50, DAX, CAC 40, FTSE 100, FTSE MIB, IBEX 35, AEX
\item Asia Ex-Japan: Shanghai Comp, Hang Seng, KOSPI, ASX 200, NZX 50
\item Japan: Nikkei, Topix, JASDAQ
\item Emerging Markets: Bovespa, NIFTY, Micex/RTS, Mexico IPC, TOP 40/JALSH, BIST 30/100 etc.
\end{enumerate}
Indices can be traded directionally or as relative value trades (long one index, short another). Different indices have different sector exposures: Bovespa is more commodity-sensitive than NIFTY; S\&P 500 has more financial exposure than Nasdaq (technology-weighted).
\end{remark}

\begin{remark} \hlt{Index Construction Methods}\\
Index construction methodology affects interpretation:
\begin{enumerate}[label=\roman*.]
\setlength{\itemsep}{0pt}
\item Price-Weighted: considers only stock price, ignores market cap. A single large stock move can dominate.
\item Market-Cap Weighted: weighted by market capitalisation. Large-cap stocks dominate performance.
\item Equal-Weighted: each constituent has equal weight regardless of size.
\end{enumerate}
Understanding weighting structure is essential for accurate interpretation of index moves.
\end{remark}

\begin{remark} \hlt{Sector Rotation Across Economic Cycles}\\
Different sectors outperform in different economic regimes:
\begin{enumerate}[label=\roman*.]
\setlength{\itemsep}{0pt}
\item Recession/Contraction: Consumer Staples, Utilities, Healthcare (defensive sectors)
\item Recovery: Financials, Consumer Discretionary, Real Estate
\item Expansion: Industrials, Materials, Technology
\item Slowing Growth/Late Cycle: Energy, Materials (inflation hedges)
\end{enumerate}
To best position an equity portfolio, it is critical to know whether the economy is in contraction, recovery, expansion, or slowing growth.
\end{remark}

\begin{remark} \hlt{Equity Derivatives Overview}\\
Key equity derivatives for macro trading:
\begin{enumerate}[label=\roman*.]
\setlength{\itemsep}{0pt}
\item ETFs: fast, cheap exposure to indices, sectors, countries, commodities, FX, FI. Trade like stocks.
\item ADRs (American Depositary Receipts): foreign stocks listed on U.S. exchanges in USD. Subject to FX risk since underlying shares are held locally.
\item Index Futures: liquid, leveraged exposure to equity indices.
\item Options: provide asymmetric payoffs for hedging and directional views.
\end{enumerate}
\end{remark}

\begin{remark} \hlt{The Volatility Index (VIX)}\\
The VIX measures implied volatility of S\&P 500 options one month out. Key properties:
\begin{enumerate}[label=\roman*.]
\setlength{\itemsep}{0pt}
\item Negatively correlated with equity returns (volatility rises when equities fall)
\item During large selloffs, VIX can move from mid-teens to 30+ in days
\item VIX options provide asymmetric hedges for equity portfolios
\item VIX futures typically trade in contango; during sharp selloffs, shift to backwardation
\end{enumerate}
Volatility is measured relative to $\sqrt{252} \approx 16$. A VIX of 16 implies approximately $1\%$ expected daily move; VIX of $24$ implies $1.5\%$ daily move $(24/16 = 1.5\%)$.\\
V2X (Euro Stoxx), VXN (Nasdaq), VXD (DJIA), RVX (Russell 2000), GVZ (Gold), OVX (Crude Oil).
\end{remark}

\begin{remark} \hlt{Variance Swaps}\\
Variance swaps are OTC products providing pure variance exposure. Advantages over options:
\begin{enumerate}[label=\roman*.]
\setlength{\itemsep}{0pt}
\item Pure volatility exposure without delta hedging
\item Priced with realised volatility (usually lower than implied)
\item No interest rate or dividend risk
\end{enumerate}
Payoff structure is convex and nonlinear. Long variance benefits disproportionately from large moves in either direction. Quoted by strike (reference realised volatility), vega notional, variance units, maturity.\\
Variance units $= \frac{\text{Vega Notional}}{2 \times \text{Strike}}$, Payoff $= (\sigma_{\text{realised}}^2 - K^2) \times \text{Variance Units}$
\end{remark}

\begin{remark} \hlt{Dividend Swaps}\\
An OTC or exchange-traded product that allows one to take a view on dividends of an index to be higher, or lower, than a fixed amount. 
\end{remark}

\begin{remark} \hlt{Equity Valuation Techniques for Macro}
\begin{enumerate}[label=\roman*.]
\setlength{\itemsep}{0pt}
\item Price-to-Book (P/B): P/B $> 2.5\times$ generally overbought; P/B $< 1.5\times$ oversold. Useful for detecting bubbles and value opportunities.
\item Dividend Yield: dividend/price. Holder gets paid while waiting (long carry). Useful for relative value across markets.
\item Price-to-Earnings (P/E): price per \$1 of earnings. Best for comparable analysis across countries, sectors, stocks. Forward P/E uses analyst EPS estimates.
\item Free Cash Flow Yield: FCF/Price. Cash flow from operations minus capital expenditures. Graham and Dodd value investors use this as top screen.
\item Market Cap to GDP: useful for locating potential bubbles. Above 100\% warrants caution; rapid rises to 150\%+ indicate bubble conditions (Japan 1989).
\end{enumerate}
\end{remark}

\begin{remark} \hlt{Leading Indicators for Equities}\\
Useful macro indicators that lead equity performance:
\begin{enumerate}[label=\roman*.]
\setlength{\itemsep}{0pt}
\item PMI (Purchasing Managers' Index): leading indicator scored 0-100. Sharp slope changes and deviations from 50 provide strong signals. ISM below 50 typically indicates U.S. recession.
\item Baltic Dry Index: shipping prices for raw material dry bulk. Supply is fixed (ships take years to build), demand is inelastic. Leading indicator for raw material demand.
\item CDX High Yield Index: CDS spread on high-yield corporates. Highly correlated with VIX; inversely related to equity prices. Rising spreads signal risk-off.
\item Consumer Confidence: in the U.S., consumption accounts for $>70\%$ of GDP. Higher confidence leads to higher spending and GDP growth.
\item Commodity Prices (YoY): rising commodity prices are inflationary, typically met by central bank tightening, which has bearish equity implications.
\end{enumerate}
\end{remark}

\begin{remark} \hlt{AUD Volatility as Risk Barometer}\\
The Australian dollar is a main risk-on currency due to Australia's commodity exports and reliance on Asian economies. AUD moves in tandem with equity prices.\\
Three-month AUD/USD implied volatility is highly correlated with VIX and inversely correlated with S\&P 500. In times of stress, the rate of change in AUD volatility shifts suddenly and aggressively.\\
Using VIX and AUD volatility combined on an absolute basis smooths out error as a more useful measure.
\end{remark}



