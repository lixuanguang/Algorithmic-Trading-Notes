\subsection{Transmission and Cross-Asset Mapping}

\subsubsection{Building Blocks of Global Macro}

\begin{remark} \hlt{Macro Strategy Product Groups}
\begin{enumerate}[label=\roman*.]
\setlength{\itemsep}{0pt}
\item Fixed Income (Policy and Growth Sensor): encodes growth expectations, inflation expectations, central bank credibility. Yield curves summarise the entire macro narrative; Steepening means growth/reflation; Flattening or inversion means slowdown/recession.\\
Fixed income tells what the market thinks will happen before it happens
\item Foreign Exchange (Adjustment Mechanism): FX reflects relative, not absolute, macro conditions. Sensitive to rate differentials, capital flows, balance of payments, risk-on/risk-off regimes.\\
FX absorbs and redistributes macro shocks across countries.
\item Equities (Growth Expression): equities are long growth, long liquidity, short uncertainty. Equity indices aggregate earnings expectations, discount rates, risk appetite.\\
Volatility (VIX) is a second-order macro signal, not just an equity signal.\\
Equities express the confidence (or fear) embedded in the macro outlook.
\item Commodities (Inflation and Real-Economy Link): driven by physical supply and demand, inventories, geopolitics. Highly sensitive to growth cycles, inflation shocks, currency moves (especially USD).\\
Commodities anchor macro views in real economic constraints.
\end{enumerate}
\end{remark}

\begin{remark} \hlt{Relationship Between Assets}\\
These relationships are regime-dependent, not permanent laws.
\begin{enumerate}[label=\roman*.]
\setlength{\itemsep}{0pt}
\item Growth Shocks: rates rise, equities rally, cyclicals outperform, commodities rise. FX favours growth-linked currencies (AUD, CAD, EM FX etc.).
\item Recession Shocks: rates fall, curves flatten/invert, equities sell off, USD strengthens, commodities weaken (except gold)
\item Inflation Shocks: nominal rates up, real rates determine equity impact. Commodities outperform. FX punishes low-credibility central banks
\end{enumerate}
\end{remark}

\begin{remark} \hlt{Asset Correlation Relationships}\\
Asset correlations change under stress. Diversification fails when you need it most. Assets that appear uncorrelated in calm periods may collapse together in crises.\\
Macro traders profit when they anticipate correlation breakdowns, and position for nonlinear regime shifts.\\
This is why macro portfolios use convex instruments, size conservatively, focus on liquidity.
\end{remark}

\begin{remark} \hlt{Macro View Thesis}\\
A macro view expressed in only one asset class is incomplete.\\
A single macro thesis should have multiple expressions, each with different risk profiles, for example
\begin{enumerate}[label=\roman*.]
\setlength{\itemsep}{0pt}
\item Growth view: rates, equities, FX
\item Inflation view: commodities, breakevens, FX
\item Crisis view: bonds, USD, volatility
\end{enumerate}
\end{remark}

\newpage

\subsubsection{Foreign Exchange in Global Macro}

FX plays three special roles simultaneously:
\begin{enumerate}[label=\roman*.]
\setlength{\itemsep}{0pt}
\item Macro shock absorber 
\item Macro transmission channel 
\item Macro expression instrument
\end{enumerate}

\begin{remark} \hlt{Role of the US Dollar}\\
Global macro is de facto USD-centric, as USD is the global reserve currency, most commodities are USD-priced, global funding markets clear in USD, and cross-border leverage is USD-denominated.\\
Many “local” macro trades are actually USD liquidity trades in disguise.\\
Stress often appears first in US Dollar Index (DXY), cross-currency basis, EM FX before local assets.
\end{remark}

\begin{remark} \hlt{Currency Regimes}\\
Currency regimes matter more than models. FX does not behave uniformly. Regime dominates signal.
\begin{enumerate}[label=\roman*.]
\setlength{\itemsep}{0pt}
\item Free-floating regimes: price discovery happens in FX
\item Managed/pegged regimes: pressure builds elsewhere (rates, reserves, capital controls)
\item Hard pegs: FX looks stable until it breaks violently
\end{enumerate}
FX valuation only works within regime constraints. Ignore the regime and “cheap” can stay cheap forever.
\end{remark}

\begin{remark} \hlt{Valuation Techniques}
\begin{enumerate}[label=\roman*.]
\setlength{\itemsep}{0pt}
\item Market Implied and Risk-Appetite Signals
\begin{enumerate}[label=\arabic*.]
\setlength{\itemsep}{0pt}
\item Equity Index Price Performance: strong local equities (vs peers) attract inflows and support currency; sharp equity underperformance foreshadow FX weakness via risk-off deleveraging and foreign selling.
\item Credit/Sovereign Risk: widening sovereign spreads/CDS is a direct “default/policy credibility/funding stress” signal weaken FX or raise devaluation risk premia). Also a regime detector for EM FX.
\item Sentiment/Positioning: call-put skew on three-month risk reversals. Acts as a multiplier. Can create overshoots/undershoots around fair value in crowded carry or crisis episodes.
\end{enumerate}
\item Trade and Competitiveness Fundamentals
\begin{enumerate}[label=\arabic*.]
\setlength{\itemsep}{0pt}
\item Trade Balance/External Account: structurally strong trade position often supports FX; persistent deficits can make the currency reliant on financing
\item Trade-Weighted Index (TWI): reflects currency’s effective competitiveness against trading partners currency basket that matters for the real economy.
\item Economic Activity: strong activity can support FX via higher expected returns and inflows, but it can also worsen the trade balance
\item Export Partner Growth and Direct Exports: mapping who buys exports and whether those buyers are expanding/contracting. Export partner growth leads exporter’s FX through expected trade receipts.
\end{enumerate}
\item Macro Sustainability/Macro Indicators: 
\begin{enumerate}[label=\arabic*.]
\setlength{\itemsep}{0pt}
\item Debt-to-GDP: high public debt with FX liabilities or weak domestic savings can increase risk premia and currency vulnerability, as it tightens the policy constraint set (inflation vs austerity vs default).
\end{enumerate}
\item Long-Term Valuation Anchors
\begin{enumerate}[label=\arabic*.]
\setlength{\itemsep}{0pt}
\item Absolute and Relative PPP: can take up to several years for consumer prices to converge.
\item GDP Per Capita: structural development/productivity proxy. Richer, higher productivity economies tend to sustain stronger real exchange rates over time.
\end{enumerate}
\item Carry Indicators
\begin{enumerate}[label=\arabic*.]
\setlength{\itemsep}{0pt}
\item 
\end{enumerate}it
\end{enumerate}
\end{remark}







