\subsection{Practitioner Frameworks}

This section distills operational frameworks from leading global macro practitioners, drawn primarily from Steven Drobny's interviews (\cite{drobny_2006}). Focus is on actionable principles rather than historical narratives.

\subsubsection{Portfolio Construction and Risk Frameworks}

\begin{remark} \hlt{Systematic Risk Premia Allocation (Leitner Framework)}
\begin{enumerate}[label=\roman*.]
\setlength{\itemsep}{0pt}
\item Equal $20\%$ risk allocation across: Equities (value screens), Fixed Income ($5$x leveraged short-duration), FX (carry + vol premia), Commodities (roll yield), Real Estate
\item Additional $10\%$ for $1$--$2$ high-conviction opportunities annually
\item Anchor to data first, then narratives; actively seek disconfirming evidence
\item Not being invested = foregoing risk premia; short positions pay away premia
\end{enumerate}
\end{remark}

\begin{remark} \hlt{Convexity and Optionality (Siva-Jothy/Anonymous Currency Manager)}
\begin{enumerate}[label=\roman*.]
\setlength{\itemsep}{0pt}
\item Maintain long gamma; never short gamma in macro portfolios
\item $50$--$60\%$ options ($3$--$6$ months); $75\%$ of price moves occur $10$--$15\%$ of time around data releases
\item Systematically long options to capture regime shifts; leverage is function of correct macro analysis
\item Never forced seller; liquidity-only mandate (G10 only)
\item Timing: buy options when consensus says inevitable and market at extreme
\end{enumerate}
\end{remark}

\begin{remark} \hlt{Trade Construction Discipline (Drobny/Wadhwani)}
\begin{enumerate}[label=\roman*.]
\setlength{\itemsep}{0pt}
\item Views without actionable trades are useless; compare fundamentals to market perception
\item Look for positioning extremes ($2$--$5$ year timeframes); CA deficit currency rising = longs building
\item Minimum risk/reward $4:1$; after two stop-outs on same thesis, move on
\item $3$--$4$ major opportunities annually; overtrading as damaging as running losers
\item Positioning and sentiment dominate fundamentals short-term
\item Stop-loss most critical; exit immediately when thesis invalidated
\end{enumerate}
\end{remark}

\begin{remark} \hlt{Fixed Income Specific (Porter)}
\begin{enumerate}[label=\roman*.]
\setlength{\itemsep}{0pt}
\item Time horizon arbitrage: institutional short-termism creates medium-term opportunity
\item Focus on one-year, one-year forwards (most volatile, greatest overshooting)
\item Interest rate cycles $16$--$24$ months; maximise income per VaR via front-end
\item Curve steepener $\approx 90\%$ bullish bonds; flattener $\approx$ bearish
\end{enumerate}
\end{remark}

\subsubsection{Asset-Class Specific Frameworks}

\begin{remark} \hlt{Commodity Framework (Anderson)}
\begin{enumerate}[label=\roman*.]
\setlength{\itemsep}{0pt}
\item Pure microeconomics: supply/demand, cost curves, competitive positioning
\item Fundamentals over price: ignore day-to-day moves, focus on economics
\item Mean reversion: margins revert to mean, not prices
\item Contrarian timing: front-page hysteria = reversal indicator (sell euphoria, buy panic)
\item Micro-economic inevitability: stress-test thesis against adverse macro scenarios. Long = assume low growth; short = assume high growth
\item On-the-ground research: observe anecdotal evidence, see fundamental change before markets price it
\item Only invest low-cost producers; high-cost assets can't be bought cheap enough
\item Reject VaR for concentrated portfolios (assumes diversification); use volatility-adjusted sizing
\end{enumerate}
\end{remark}

\begin{remark} \hlt{Equity Macro Framework (Bessent)}
\begin{enumerate}[label=\roman*.]
\setlength{\itemsep}{0pt}
\item Express macro views through equities: $50\%$ of stock move = market, $30\%$ = sector, $20\%$ = stock-specific
\item Bottom-up research for macro confirmation: micro insights reveal macro picture
\item Entry points critical: good entry means rarely stopped out
\item Concentrated portfolio: $8$--$14$ large positions, not equally weighted
\item Paradigm shifts = major profits: identify when game changes, not how to play game better
\end{enumerate}
\end{remark}

\begin{remark} \hlt{FX Framework (Anonymous Currency Manager)}
\begin{enumerate}[label=\roman*.]
\setlength{\itemsep}{0pt}
\item FX is tail of credit curve: Fed funds $\rightarrow$ short rates $\rightarrow$ govts $\rightarrow$ credit $\rightarrow$ equities $\rightarrow$ FX
\item Currencies reflect relative sentiment/credit spreads between economies
\item Three currency blocs: USD, EUR, Asia
\item Directional concentration: $3$--$4$ macro calls maximum (bonds, equities, USD vs blocs)
\item Five-month time horizon for thesis validation
\item Basic law: bullish = long or flat, bearish = short or flat. Never relative value
\end{enumerate}
\end{remark}

\begin{remark} \hlt{Emerging Markets Crisis Framework (Dimitrijević)}
\begin{enumerate}[label=\roman*.]
\setlength{\itemsep}{0pt}
\item Country implosion acts like company collapse: all instruments correlate
\item Strict limits: $10\%$ global macro fund, $15\%$ EM fund per country
\item Best EM trades after panics/disasters
\item Clean sheet valuation: ignore yesterday's price. Ask: ``would we invest at current price with clean sheet?''
\item Stop-loss mechanics: work in continuous markets, dangerous in distressed/value plays
\item Internal stops only (counterparties invariably hit external stops)
\item One-month cool-off after stop-out
\item Hedge mismatch: short rich asset class as hedge to long attractive asset class within same country
\end{enumerate}
\end{remark}

\subsubsection{Execution and Flow Analysis}

\begin{remark} \hlt{Flow and Cross-Market Linkages (Harris)}
\begin{enumerate}[label=\roman*.]
\setlength{\itemsep}{0pt}
\item Flow matrix: mental model tracking global money flows seeking highest risk-adjusted return
\item Money craves stability; fear greater than greed drives capital flight
\item Cross-market linkages: geopolitical events ripple through asset classes (e.g., Brazilian real devaluation $\rightarrow$ sell soybeans as farmers hedge USD revenues)
\item Slower Fool Theory: exit before discontinuous leap into illiquidity
\item Risk management: $80\%$ trades wrong, $20\%$ profitable must exceed losses
\item Small probes (reconnaissance) test thesis before scaling
\item Never pray trades back to breakeven; clear books, start fresh
\item Best at $3$--$4$ concentrated positions; avoid overtrading
\end{enumerate}
\end{remark}

\begin{remark} \hlt{Contrarian Indicators}
\begin{enumerate}[label=\roman*.]
\setlength{\itemsep}{0pt}
\item Concept exhaustion: magazine covers = fully discounted
\item Gestalt shifts: volatility spikes when protection buying signals crowded positioning
\item ``Frozen ice theory'': strong trades at edges, crowded in middle on proxies
\item Central banks often counter-indicators in market timing
\end{enumerate}
\end{remark}

