\subsection{Practitioner Perspectives on Global Macro}

This section draws from ``Inside the House of Money: Top Hedge Fund Traders on Profiting in the Global Markets'' by Steven Drobny (\cite{drobny_2006}), a collection of interviews with legendary global macro practitioners.

\subsubsection{Historical Case Studies in Global Macro}

\begin{remark} \hlt{Stock Market Crash of 1987}\\
Black Monday introduced the concepts of liquidity risk and fat tails to the wider investment community:
\begin{enumerate}[label=\roman*.]
\setlength{\itemsep}{0pt}
\item Portfolios were obliterated as margin calls went unfunded
\item ``Portfolio insurance'' hedges failed as futures and options markets became unhinged from the cash market
\item Commodity-stream traders (e.g., Paul Tudor Jones) profited by identifying technical similarities to 1929
\item Equity-stream traders (e.g., Soros) took significant losses but still ended the year positive
\end{enumerate}
The crash marked Alan Greenspan's first major test as Fed chairman, initiating the ``Greenspan put'', which is an implicit option that the Fed writes anytime equity markets stumble.
\end{remark}

\begin{remark} \hlt{Black Wednesday 1992 (ERM Crisis)}\\
The sterling crisis of September 16, 1992, brought ``global macro'' into the public vocabulary:
\begin{enumerate}[label=\roman*.]
\setlength{\itemsep}{0pt}
\item The UK joined the ERM in 1990 at an arguably overvalued parity rate of 2.95 DEM/GBP
\item Strong currency during recession forced artificially high interest rates to maintain the peg
\item Speculators sold pounds against Deutsche marks, forcing Bank of England intervention
\item Rate hikes from 10\% to 12\% (with threats of 15\%) emboldened rather than discouraged traders
\item Traders knew that raising rates to defend a currency during recession is unsustainable
\item The pound fell approximately 15\% after withdrawal from ERM
\end{enumerate}
Key insight: Policy inconsistency creates opportunity. Traders profit when policymakers attempt to maintain incompatible objectives (fixed FX + independent monetary policy + free capital flows).
\end{remark}

\begin{remark} \hlt{Bond Market Rout of 1994}\\
The 1994 bond massacre demonstrated the dangers of leveraged carry trades:
\begin{enumerate}[label=\roman*.]
\setlength{\itemsep}{0pt}
\item After keeping rates at 3\% through 1993, the Fed began tightening in February 1994
\item Long-duration bond positions, heavily leveraged by hedge funds and dealers, were devastated
\item 10-year Treasury yields rose from 5.6\% to over 8\% by November 1994
\item Estimated losses exceeded \$1.5 trillion globally across bond markets
\item Orange County, California declared bankruptcy after losing \$1.7 billion on leveraged inverse floaters
\item Many macro funds were caught positioned for continued low rates
\end{enumerate}
Key lesson: Central bank regime shifts create violent repricing. Positions that rely on stable monetary policy are implicitly short volatility. When the Fed pivots, duration risk materialises instantly.
\end{remark}

\begin{remark} \hlt{Asia Crisis 1997}\\
The Asian financial crisis demonstrated how fixed exchange rate regimes can unravel:
\begin{enumerate}[label=\roman*.]
\setlength{\itemsep}{0pt}
\item Overvalued pegged currencies encouraged excessive foreign-currency borrowing
\item When Thailand abandoned its peg in July 1997, contagion spread across Southeast Asia
\item The ``Asian flu'' moved from Thailand to Indonesia, Malaysia, Korea, and beyond
\item Malaysia imposed capital controls and pegged the ringgit at 3.80 per USD
\item Local borrowers with unhedged FX exposure were as culpable as speculators
\end{enumerate}
The crisis highlighted that speculators do not drive markets, they exploit underlying policy inconsistencies.
\end{remark}

\begin{remark} \hlt{Russia/LTCM Crisis 1998}\\
The Russian default and LTCM collapse taught critical lessons about correlation and liquidity:
\begin{enumerate}[label=\roman*.]
\setlength{\itemsep}{0pt}
\item Russia devalued the ruble and defaulted on 281 billion rubles of government debt in August 1998
\item Flight to quality caused spreads between on-the-run and off-the-run Treasuries to blow out
\item LTCM's models calculated a 1-in-6-billion chance of blowup, but ignored the ``LTCM liquidity premium'', which are positions correlated solely because they were all in LTCM's portfolio
\item Liquidity is never there when really needed
\item There is no such thing as risk-free arbitrage when leverage is required
\end{enumerate}
Risk management systems based on historical prices have critical blind spots. Fat tail events are more frequent than normal distributions suggest.
\end{remark}

\begin{remark} \hlt{Dot-Com Bust 2000}\\
The tech bubble collapse marked the end of the global macro mega-fund era:
\begin{enumerate}[label=\roman*.]
\setlength{\itemsep}{0pt}
\item Greenspan's ``irrational exuberance'' speech (December 1996) was followed by three more years of rally
\item S\&P 500 doubled while NASDAQ quadrupled from 1996 to March 2000
\item Macro managers struggled with the momentum-driven paradigm
\item Tiger Management closed after maintaining a losing ``long old economy / short new economy'' thesis
\item Soros flipped from short to long tech stocks in late 1999, then suffered heavy losses when the bubble burst
\end{enumerate}
Ironically, Tiger's final month (March 2000) coincided with the absolute top of the tech bubble. The crisis reinforced that being right and being early can be indistinguishable from being wrong.
\end{remark}

\subsubsection{Evolution from Global Macro to Global Micro}

\begin{remark} \hlt{Specialization Within Macro}\\
The global macro landscape has evolved since the mega-fund era:
\begin{enumerate}[label=\roman*.]
\setlength{\itemsep}{0pt}
\item No single fund dominates markets as Soros and Tiger once did
\item Today's managers derive edge through micro expertise expressed via the macro mandate
\item Large macro complexes (Tudor, Caxton) allocate to internal teams with specific skills
\item Diversification occurs across strategies within the fund, not just across positions
\item ``Global macro only means that you start at the top and work your way down''
\end{enumerate}
The approach: form macro views on countries and regimes, then drill down to sectors and companies that benefit from those views. Macro themes expressed in a micro style.
\end{remark}

\begin{remark} \hlt{Sources of Edge in Modern Macro}\\
Where practitioners find opportunities today:
\begin{enumerate}[label=\roman*.]
\setlength{\itemsep}{0pt}
\item \textbf{On-the-ground research}: Travel, meet management, understand local conditions
\item \textbf{Network effects}: Information flows from relationships built over decades
\item \textbf{Patience}: Ability to wait for the right opportunity without forcing trades
\item \textbf{Flexibility}: Willingness to express views across any asset class or instrument
\item \textbf{Contrarian positioning}: Buy value in overlooked markets
\end{enumerate}
\end{remark}


\subsubsection{Portfolio Manager Notes}

\begin{remark} \hlt{Jim Leitner (Falcon Management): The Family Office Manager}\\
Jim Leitner ran proprietary trading at Bankers Trust in the 1980s, where he became a pioneer in systematic carry trading. After founding Falcon Management in 1997, he returned approximately $30\%$ compound annually. His approach combines family office time horizons with hedge fund discipline.\\

\textbf{Career Arc and Skill Development}
\begin{enumerate}[label=\roman*.]
\setlength{\itemsep}{0pt}
\item Started as a money broker trainee at Dominion Securities while in graduate school
\item Traded interbank Eurodollar deposits, learning the plumbing of the financial system
\item Moved to JP Morgan Nassau desk (salary dropped from $\$70,000$ to $\$17,000$ as an investment in learning)
\item Became expert in ``esoteric'' currencies (pesetas, bolivar, peso), markets where information edge mattered
\item Rose to primary risk taker at Bankers Trust during its golden era of prop trading (1986--1991)
\item Calculated lifetime P\&L: extracted over \$2 billion from markets for employers and investors
\end{enumerate}
Key insight: Expertise comes from obsessive interest and continuous exposure, not innate talent.\\

\textbf{Psychological Edge}: Leitner identifies two psychological traits that distinguish successful traders:
\begin{enumerate}[label=\roman*.]
\setlength{\itemsep}{0pt}
\item \textbf{Unemotional about numbers}: losses are viewed as purely probability-driven
\item \textbf{Humble about ignorance}
\end{enumerate}

\textbf{Fighting Confirmation Bias}: There is need to seek disconfirming evidence. Humans are preprogrammed to look for confirmatory evidence, hence train to ask why we might be wrong. Stories make trades emotionally appealing but can override quantitative discipline; always anchor to numbers first, then look for a story.\\

\textbf{Options as Risk Management}: Options are extensively used as a risk management tool. Long volatility positions allow peace of mind. The only way to hedge the unknown is to cut off tail risk completely.\\

\textbf{Five Asset Class Framework}: Falcon allocates 20\% risk to each of five asset categories:
\begin{enumerate}[label=\roman*.]
\setlength{\itemsep}{0pt}
\item \textbf{Equities}: Seek value stocks in undervalued countries. Use quantitative screens (P/B, P/E, P/CF) first, then look for stories. Example: Ghanaian stocks up several hundred percent; Guinness Nigeria after beer consumption rebounded from 3L to 4L (down from 34L).
\item \textbf{Fixed Income}: Use leverage to earn risk premia. Short-duration bonds have highest Sharpe ratios. Can be 5x levered in 2-year bonds with same risk as unlevered 10-year bonds.
\item \textbf{Foreign Exchange}: Systematically extract carry and volatility premia. Weekly purchase of one-year straddles across five currency pairs (12.5 bps premium/week, 6.5\% NAV annually).
\item \textbf{Commodities}: Roll futures contracts to capture risk premia (effectively insuring commodity producers). Historical returns: approximately 6\% real per annum.
\item \textbf{Real Estate}: Buy real estate-linked equities globally rather than physical property. Holdings in Poland, Finland, Sweden, Spain, Hong Kong, Indonesia, Singapore, Taiwan.
\item \textbf{Absolute Return} bucket: 10\% allocation for ``home run'' ideas that appear 1--2 times per year. Examples: 5-year gold puts at \$3 (sold at \$18), post-1998 Russian equities, Icelandic inflation-linked housing bonds yielding 5\% real.
\end{enumerate}

\textbf{Global Macro to Global Micro}: The global macro style of Leitner works as follows:
\begin{enumerate}[label=\roman*.]
\setlength{\itemsep}{0pt}
\item Start by differentiating good countries (twin surplus) from bad countries (twin deficit)
\item For good countries: dig in to understand what is changing for the better
\item Drill down to sectors and companies that benefit from macro view
\end{enumerate}

\textbf{Information Sourcing and Research Process}: Leitner's systematic approach to finding opportunities:
\begin{enumerate}[label=\roman*.]
\setlength{\itemsep}{0pt}
\item Read \textit{The Economist} religiously, and think about how to develop knowledge learnt into a trade
\item Develop a network by meeting intelligent people, don't be afraid to be ignorant or ask questions
\item When you find a compelling idea you don't know much about, put a tiny amount of money into it and treat it like a cheap option
\item Travel extensively: visit central banks, meet management, understand local conditions
\end{enumerate}

\textbf{The Case for Being Invested}: Leitner's long-term philosophy aligns with endowment-style investing:
\begin{enumerate}[label=\roman*.]
\setlength{\itemsep}{0pt}
\item Not being invested means not earning risk premia
\item Running short positions means paying away risk premia; must be doubly right
\item Value stocks outperform growth stocks because growth is priced too expensively
\item Mean reversion works in the medium term
\item Disasters and tragedies are often buying opportunities
\end{enumerate}

\end{remark}


\begin{remark} \hlt{Christian Siva-Jothy (SemperMacro): The Prop Trader}\\
Christian Siva-Jothy spent 17 years as a bank proprietary trader, including 14 years at Goldman Sachs where he became partner in charge of fixed income and currency proprietary trading. His first day on a trading desk was Black Monday 1987. After retiring from Goldman, he founded SemperMacro.\\

\textbf{Career Formation at Goldman Sachs}: Siva-Jothy joined Goldman's prop desk in 1991 from Citibank. The culture shock was immediate:
\begin{enumerate}[label=\roman*.]
\setlength{\itemsep}{0pt}
\item At Goldman, to be noticed, P\&L had to be a lot higher than at CitiBank
\item First week: put on 50 million DEM/CHF position (biggest of his career to that point). Head of trading walked over: ``I like a guy who averages into his positions''
\item Early 1990s Goldman had no VAR system, no formal limit structures
\item Everyone became a prop trader; 500+ prop traders at GS, often all in the same position
\end{enumerate}

\textbf{The 1994 Sterling/Yen Disaster}: 1994 was the turning point in Siva-Jothy's career:
\begin{enumerate}[label=\roman*.]
\setlength{\itemsep}{0pt}
\item Had phenomenal years in 1992 and 1993 (over \$100 million profit in 1993)
\item Built largest position ever: over 1 billion long sterling/yen via cash and options
\item Was also selling sterling/yen puts whenever cross corrected (short gamma)
\item Clinton attacked Japan on trade policy; dollar/yen gapped 5\% lower with nothing trading
\item Five days later, awful UK inflation numbers; sterling went into free fall
\item Position dropped 10\% in weeks; lost approximately \$40 million in one day
\item Instead, took a deep breath and liquidated everything
\item Total loss: \$80--100+ million of Goldman's money
\end{enumerate}
Post-1994, GS dramatically restructured prop trading: 20+ pure prop traders reduced to 3 by mid-1995. Siva-Jothy was asked to rebuild with a very different risk mandate.\\

\textbf{Key Lessons from 1994}
\begin{enumerate}[label=\roman*.]
\setlength{\itemsep}{0pt}
\item Overconfidence is an absolute killer
\item Never be short gamma: Portfolios in the macro space should always be long gamma
\item Liquidity comes at a cost
\item Use options for tail protection
\end{enumerate}

\textbf{1998: The Reverse of 1994}. The Russia/LTCM episode validated lessons learned:
\begin{enumerate}[label=\roman*.]
\setlength{\itemsep}{0pt}
\item GS had a risk arbitrage group of young PhDs with relatively little trading experience
\item Watched colleagues make the same mistakes: overconfident from prior years' success
\item When it unraveled, Siva-Jothy's group loaded up on Eurodollar contracts
\item 1998 was their best year since 1994, but swamped by risk arb losses
\end{enumerate}

\textbf{The Go-To Trade: Long Eurodollars}. Siva-Jothy's approach to dislocations:
\begin{enumerate}[label=\roman*.]
\setlength{\itemsep}{0pt}
\item When there is dislocation in the direction of pain to the U.S. economy, buy Eurodollars
\item Preference for liquid markets: When the Fed is in an aggressive rate-cutting mode, long Eurodollars
\item Long fixed income if there is a major dislocation to the economy; this is a synthetic long gamma trade
\item Bear markets in fixed income are short with powerful rallies; bull markets are where medium-term players make money
\end{enumerate}

\textbf{September 11, 2001 Trade}. A formative example of event-driven trading:
\begin{enumerate}[label=\roman*.]
\setlength{\itemsep}{0pt}
\item Already long U.S.\ fixed income on structural view that economy was weak
\item Heard about first plane; noticed perfectly clear blue sky day. Terrorist act $\to$ whack consumer sentiment
\item Bought Eurodollars and calls on Eurodollars after first plane but before second
\item Markets rallied only 13 bps on the day
\end{enumerate}

\textbf{Favourite Trades}
\begin{enumerate}[label=\roman*.]
\setlength{\itemsep}{0pt}
\item \textbf{Bund/BTP Convergence (1995--1999)}: Believed in political will for European monetary union. BTP spread to Bunds was 500bp over; structured currency barrier range bet option with compressing knockout feature. Ran trade from 500 over to 50 over.
\item \textbf{ERM Exit (1992)}: Made little on short sterling bet, but bought calls on short sterling futures when UK raised rates to 10\% then 13\% to defend pound. Raising rates when the UK economy is going down is bad move; implied volatilities are always massively undervalued during big events.
\end{enumerate}

\textbf{Trading Process and Philosophy}
\begin{enumerate}[label=\roman*.]
\setlength{\itemsep}{0pt}
\item Price action is paramount
\item Trading diary discipline: Every morning review positions, What has changed.
\item Time horizon: Medium-term (zero to 3--4 months), but constantly reassessing
\item Portfolio structure: 50--60\% options (3--6 month maturity), plus overnight options; cash opportunistic
\item Skepticism of travel for research: to know a place really well for there to be value in going back.
\end{enumerate}

\textbf{Three to Four Major Opportunities Per Year}\\
Catch one, high single-digit returns. Catch two, 10--20\% returns. Catch three or four, doing incredibly well.\\
Overtrading is as bad as running losing positions for too long.\\

\textbf{Psychological Edge}\\
Humility, emotional stability, no blame culture, poor memory as advantage (not influenced by past events).\\

\textbf{Hedge Fund vs.\ Bank Prop Trading}
\begin{enumerate}[label=\roman*.]
\setlength{\itemsep}{0pt}
\item Banks view prop as a call option/hedge for when franchise does poorly
\item During tightening cycles and credit events, bank wants prop to make outsized returns
\item At GS, never had uncertain payout, which kept traders pushing the envelope
\item Hedge funds need different structure: Siva-Jothy uses 6-month performance resets
\item Going from 30\% to 20\% is different from going 0\% to $-10$\%
\item Optimal team size: 8--9 senior traders maximum before losing team ethos
\end{enumerate}

\textbf{Hiring Philosophy}
\begin{enumerate}[label=\roman*.]
\setlength{\itemsep}{0pt}
\item Passion, humility, integrity
\item Is this a good person, someone to sit next to that can be trusted
\item Prefers experienced traders who are self-managing
\end{enumerate}

Recommended reading: \textit{Reminiscences of a Stock Operator} by Edwin Lefevre.
\end{remark}


\begin{remark} \hlt{Dr.\ Andres Drobny (Drobny Global Advisors): The Researcher}\\
Academic economist turned Wall Street strategist and proprietary trader; PhD from Cambridge, former chief economist and head of research at Bankers Trust, helped build global FX trading at Credit Suisse First Boston, now runs independent research firm Drobny Global Advisors (DGA). Known for contrarian, out-of-consensus views and facilitating dialogue among top hedge fund managers.\\

\textbf{Career Arc}: Academia to Markets
\begin{enumerate}[label=\roman*.]
\setlength{\itemsep}{0pt}
\item Entered markets at age 33 after Financial Times published his letter on the falling USD (1986)
\item Head of Research at Bankers Trust recruited him; arrived August 1987, just before the crash
\item The crash taught him that academic economics gives theoretical grounding but not market timing
\item Became head of research in May 1989; fell in love with markets
\item Key realisation: Talk is cheap, views are worthless without actionable trade ideas
\end{enumerate}

\textbf{The Mark/Swiss Trade (1989)}: First Big Winner
\begin{enumerate}[label=\roman*.]
\setlength{\itemsep}{0pt}
\item Berlin Wall fell; DEM rallied against all currencies including CHF
\item Swiss authorities responded with aggressive rate hikes but market ignored it
\item Saw technical level with market positioned wrong way
\item Shorted 40 million DEM/CHF, got stopped out, re-entered
\item Cross broke technical level and fell like a stone, made \$1 million on small position
\item Lesson: Fundamentals drive the view, but find good trades where few are looking
\end{enumerate}

\textbf{ERM Breakup (1992)}: Seeing It Early
\begin{enumerate}[label=\roman*.]
\setlength{\itemsep}{0pt}
\item ERM carry trade had developed 1985--1991: fixed currency ranges meant rates adjusted instead of spot
\item High-yielding currencies were profitable buys when at bottom of bands
\item German reunification changed everything: German inflation accelerated, rates rose
\item Risk/reward of carry trade deteriorated as German rates rose and other economies needed lower rates
\item November 1991: Featured in International Herald Tribune calling for DEM revaluation
\item Bankers Trust management uncomfortable with research head talking against their carry positions
\item Lesson: Risk/reward became awful but investors stayed because ``it had been great''
\end{enumerate}

\textbf{CSFB Period (1992--1998)}
\begin{enumerate}[label=\roman*.]
\setlength{\itemsep}{0pt}
\item Joined Marc Hotimsky to build FX business; was strategist, trader, hedge fund partner
\item First trade: Long Finnish markka, a disaster. FIM devalued again in early September 1992
\item Quickly pivoted: If FIM going down twice, the whole ERM system must break
\item Went aggressively short Spanish peseta, British pound, Swedish krona
\item Additional insight: Japanese wall of money retreating home meant sell Europe vs.\ yen
\item 1992--1993 was a ``killer year'', but 1994 was difficult: Long USD, short bonds on Fed rate hikes. Caught bonds but gave it back on dollar weakness
\end{enumerate}

\textbf{Trade Construction Philosophy}
\begin{enumerate}[label=\roman*.]
\setlength{\itemsep}{0pt}
\item Use fundamentals to develop a view, but views are cheap
\item Compare fundamental view to market perception to find trades
\item Look for positioning extremes: If everyone is bearish, even if you're a bear, pause
\item Think about long-term positioning (2--5 years), not short-term overbought/oversold
\item Current account deficit currency going up = longs building; surplus currency going down = shorts building
\item Best opportunities: After big trends, when market is leaning one way and fundamentals suggest change
\end{enumerate}

\textbf{Differentiate Views from Trades}
\begin{enumerate}[label=\roman*.]
\setlength{\itemsep}{0pt}
\item Biases tell you directional view; trades attempt to extract value with good risk/reward
\item If bias is set, only allow positions consistent with bias or flat
\item If move to neutral bias, can trade both directions
\item Discipline: Forces deep thinking about views underlying each trade
\item If views are confused, back off on trades
\end{enumerate}

\textbf{Favourite Trade Example: Long EUR/USD (April 2002)}
\begin{enumerate}[label=\roman*.]
\setlength{\itemsep}{0pt}
\item Had bearish USD view in 2001 but was wrong. Dollar held up despite equity crash and rate cuts
\item Question was ``Why now?'' not just ``Will it happen?''
\item Hypothesis: USD strength in 2001 was due to U.S.\ companies repatriating capital during profit squeeze
\item By early 2002, companies ``flush with cash'', no longer needed to repatriate
\item USD was ``free to go down''; caught beginning of tremendous move
\end{enumerate}

\textbf{Analyst vs.\ Trader}
\begin{enumerate}[label=\roman*.]
\setlength{\itemsep}{0pt}
\item Good analyst finds the trade or identifies pitfalls, good trader manages the risk and runs the position
\item Key difference is in ability to manage large position sizes
\item Having trades on makes you profoundly honest and forces you to think ahead
\end{enumerate}

\textbf{Risk Management Philosophy}
\begin{enumerate}[label=\roman*.]
\setlength{\itemsep}{0pt}
\item Each trade is an expected return equation
\item Risk/reward should be at least 4:1; great traders think 8:1 or 10:1
\item After getting stopped out twice, what was 4:1 becomes 4:3, move on to another trade
\item Panic early in the markets. Panicking late is a recipe for disaster
\item After big losses, take a month away. Recent events impairs judgment
\item Forced vacation rule: Everyone should step away 3--4 weeks annually to clear head
\end{enumerate}

\textbf{On Speculators and Market Drivers}
\begin{enumerate}[label=\roman*.]
\setlength{\itemsep}{0pt}
\item Sterling collapsed because 10\% rates were unsustainable when economy needed 5\%
\item Speculators don't influence the trend; underlying pressures combined with policy decisions drive events
\item Sometimes speculators dampen volatility, sometimes amplify, but not the fundamental driver
\end{enumerate}

\textbf{Global Macro Portfolio Construction}
\begin{enumerate}[label=\roman*.]
\setlength{\itemsep}{0pt}
\item 80\% of capital: Carry trades and lower-risk income generators
\item 20\% of capital: Looking for 20--30\% return opportunities
\item Want diversification by trading style, not by market or geography
\item Carry trader (regular income) + Gamma trader (big moves) = diversified portfolio
\item Great trades combine carry with gamma: high-yielding currency that is also cheap
\end{enumerate}

\textbf{On Good Hedge Fund Managers}
\begin{enumerate}[label=\roman*.]
\setlength{\itemsep}{0pt}
\item Strong views but differentiate views from trades, ability to change their mind
\item Best managers: Lose small when wrong, make big when right
\item Warning signs: ``Markets were irrational'' or ``noise in the markets'' as excuses
\item Overconfidence in views is dangerous; confidence in trade risk/reward is acceptable
\item Always ask ``What if I'm wrong?''
\end{enumerate}

\textbf{Research Process}
\begin{enumerate}[label=\roman*.]
\setlength{\itemsep}{0pt}
\item Read Financial Times, The Economist, academic pieces, independent research
\item Bank research gives benchmark of consensus thinking
\item Compare reality to what people are saying, find what is missed or misread
\item Prefers analytical distance over on-the-ground research
\end{enumerate}

\end{remark}


\begin{remark} \hlt{Dr.\ John Porter (Barclays Capital): The Treasurer}\\
American with PhD in psychology (Sorbonne) and master's in economics; former chief investment officer at World Bank overseeing \$20 billion fixed income portfolio; worked at Moore Capital under Louis Bacon; became principal risk taker at Barclays Capital in 1998. Compound annual returns of 30\% gross since 1998. Known as a ``horizon trader'' with medium-term focus.\\

\textbf{Career Arc}
\begin{enumerate}[label=\roman*.]
\setlength{\itemsep}{0pt}
\item Did not start trading until age 35; was academically oriented, planned teaching career
\item World Bank treasury: one of best training grounds, traded 26 currencies, learned on the job
\item Moore Capital Paris: learned money management and leverage; European convergence trades
\item Barclays Capital: manages all consolidated interest rate risk from retail bank, mortgages, credit cards
\item Edge: medium-term time horizon that nobody else can execute
\end{enumerate}

\textbf{Time Horizon Arbitrage}
\begin{enumerate}[label=\roman*.]
\setlength{\itemsep}{0pt}
\item All investors today replicate the same short-term style (hedge funds, prop desks, real money)
\item Everyone held to parameters that don't allow volatility in earnings
\item Bank's accrual accounting: only accounts for current income (carry), not mark-to-market
\item Can hold positions through drawdowns as long as NPV stays positive
\item Do not go on crowded trades and strategies
\end{enumerate}

\textbf{Signature Trade}: Long Bunds / Short Gilts (1998--2004)
\begin{enumerate}[label=\roman*.]
\setlength{\itemsep}{0pt}
\item UK Gilts trading expensive to German Bunds due to Minimum Funding Requirement regulation
\item Positive carry of 250 basis points (Bunds financed 2.5\% cheaper than Gilts)
\item Put on in June 1998; DV01 of \pounds800,000, notional \pounds650 million
\item LTCM blowout caused trade to go against him; down \pounds60 million mark-to-market
\item During blowout, locked in 350 bps carry for two years
\item Held position six years; took off in mid-2004; made approximately \pounds180 million
\item Lesson: Staying power + understanding why trade went against you = ability to hold
\end{enumerate}

\textbf{Trading Focus: One-Year, One-Year Forwards}
\begin{enumerate}[label=\roman*.]
\setlength{\itemsep}{0pt}
\item Core expertise: where market predicts one-year rates will be in one year
\item Most volatile part of yield curve; greatest potential for large overshoots
\item Most interest rate cycles are 16--24 months; back months of Eurodollar futures have ``the meat''
\item Market often extrapolates more than what will actually occur
\item Maximise current income per unit of VAR
\item Concentrate risk in front end of yield curve, which has same VAR but much more current income
\item Prefer long volatility on average; only short vol when long front-end (natural hedge)
\item If compelling instrument captures 80--90\% of thesis, avoid distraction of multiple positions
\item Qualitative assessment: what makes each position right or wrong, then look for offsets
\end{enumerate}

\textbf{Curve Trade Example}: 2s/30s Steepener (2000)
\begin{enumerate}[label=\roman*.]
\setlength{\itemsep}{0pt}
\item Tech bubble burst; knew Fed would ease
\item 2-year notes at 6.75\%, curve inverted by 75 bps
\item Put on at absolute low of inversion; spread went from $-75$ to $+365$ bps, exited after 125 bps
\item Insight: Curve steepener = bullish bonds; curve flattener = bearish bonds (90\% directional)
\end{enumerate}

\textbf{Psychology as Trading Edge}
\begin{enumerate}[label=\roman*.]
\setlength{\itemsep}{0pt}
\item PhD in psychology is core advantage; markets are nothing but psychology
\item Stimulus $\to$ Response $\to$ Reward/Punishment framework from conditioning experiments
\item Positive reinforcement over many years requires long time to extinguish (late 1990s stocks)
\item Punishment extinguishes conditioned response faster (post-2000 crash)
\end{enumerate}

\textbf{Behavioural Finance Insights}
\begin{enumerate}[label=\roman*.]
\setlength{\itemsep}{0pt}
\item Young traders all make same mistakes; tendency to want to be with the crowd
\item Look for ``concept exhaustion'': when concept makes magazine covers, it's fully discounted
\item Same stimulus elicits monotonically decaying response until nobody cares
\item Markets can be overbought/oversold but can't trend without a new concept
\item Numbers give false sense of security; people rarely question where they come from
\end{enumerate}

\textbf{1994 Experience at Moore Capital}
\begin{enumerate}[label=\roman*.]
\setlength{\itemsep}{0pt}
\item Long UK ultra-longs and Italy when global bonds melted down
\item Capital tripled just before the crash; down 45\% of original allocation
\item Cut positions, switched to relative value, ended up positive by September
\item Lesson: Things can always get worse than you can possibly imagine
\end{enumerate}

\textbf{Best Trade}: The 110 vs 130 Bet (February 1999)
\begin{enumerate}[label=\roman*.]
\setlength{\itemsep}{0pt}
\item Post-LTCM; committee thought world ending; Porter very bearish fixed income
\item Bet: 30-year bond contract hits 110 before 130 (trading at 124.75); gave 2-to-1 odds
\item October 1999: contract printed 110.08; early 2000: hit 109.24. Won bet just before bonuses paid
\end{enumerate}

\textbf{VAR Critique}
\begin{enumerate}[label=\roman*.]
\setlength{\itemsep}{0pt}
\item Backward-looking; assumes correlations are stable
\item Risk manager wanted curve steepener before employment report to reduce VAR
\item Quantitative framework important but must overlay with common sense from experience
\end{enumerate}

Recommended reading: \textit{When Genius Failed} by Roger Lowenstein.

\end{remark}


\subsubsection{Risk Premia Extraction Strategies}

\begin{remark} \hlt{Forward Rate Bias (Carry Trade)}\\
One of the most robust risk premia in currency markets:
\begin{enumerate}[label=\roman*.]
\setlength{\itemsep}{0pt}
\item Forward exchange rates tend to overestimate changes in spot exchange rates
\item According to uncovered interest parity, forward rates should be unbiased predictors of future spot rates
\item Empirically, forwards that sell at discounts (high-yield currencies) produce positive returns on average
\item The trade: systematically long high-yielding currencies, short low-yielding currencies
\item Key risk: devaluations are digital events, the trade works until it doesn't
\item Risk management: exit when clear signals of pending devaluation emerge
\end{enumerate}
\end{remark}

\begin{remark} \hlt{Three Risk Premia in Currency Markets}
\begin{enumerate}[label=\roman*.]
\setlength{\itemsep}{0pt}
\item \textbf{High yield vs.\ low yield currencies}: The market overcompensates investors for higher yield on average. Over many years, this trade works despite occasional devaluations.
\item \textbf{Short-dated volatility premium}: Implied volatility exceeds realised volatility due to insurance premium. Systematically selling short-dated options is ``not too dissimilar to running a casino.''
\item \textbf{Long-dated options vs.\ drift}: Although implied vol exceeds realised vol, buying one-year straddles can be profitable because currency markets exhibit trending behaviour. The seller profits from delta hedging; the buyer profits from trend moving price far from strike.
\end{enumerate}
Signal-to-noise insight: Euro moved from 0.83 to 1.28 over 900 days (5 pips/day average). Daily range was 75 pips. Signal-to-noise ratio: 1:15. Short-term noise is mean-reverting; medium-term price action exhibits trend.
\end{remark}

\begin{remark} \hlt{Volatility Risk Premia}\\
Systematic approaches to extracting value from options markets:
\begin{enumerate}[label=\roman*.]
\setlength{\itemsep}{0pt}
\item \textbf{Short-dated volatility premium}: Implied volatility tends to exceed realised volatility because of an insurance premium component. Systematically selling short-dated options is akin to running a casino.
\item \textbf{Long-dated options and trend}: Although implied vol exceeds realised vol, buying one-year straddles can be profitable if trends move price far enough from the strike. The signal-to-noise ratio in FX can be 1:15 (75 pips daily range vs.\ 5 pips average daily move).
\item \textbf{Noise is mean-reverting}: Short-term price movements are largely noise; mean-reversion strategies work in the short term while trends dominate the medium-to-long term.
\end{enumerate}
\end{remark}

\begin{remark} \hlt{Commodity Roll Yield}\\
Systematic commodity futures strategies can extract risk premia:
\begin{enumerate}[label=\roman*.]
\setlength{\itemsep}{0pt}
\item Buy the second contract out, ride it up the curve, sell before expiry
\item This effectively provides insurance to commodity producers
\item Historical returns: approximately 6\% real returns per annum
\item The risk premium compensates for bearing inventory and price risk that producers wish to hedge
\end{enumerate}
\end{remark}

\subsubsection{Trading Psychology and Process}

\begin{remark} \hlt{Emotional Neutrality}\\
Successful macro traders share common psychological traits:
\begin{enumerate}[label=\roman*.]
\setlength{\itemsep}{0pt}
\item \textbf{Unemotional about outcomes}: Numbers are just numbers; a loss is information, not failure
\item \textbf{Humble about ignorance}: Despite success, never feel smarter than the market. ``All I know is every day there's the possibility that I will find a better trade than someone else by looking and searching.''
\item \textbf{Avoidance of confirmation bias}: Actively seek disconfirming evidence. Train yourself to ask why you might be wrong, not why you're right.
\item \textbf{Story skepticism}: Stories make trades emotionally appealing but can override quantitative discipline. Always anchor to numbers first, then look for a story.
\end{enumerate}
\end{remark}

\begin{remark} \hlt{Tail Risk Management}\\
Lessons from practitioners on managing extreme outcomes:
\begin{enumerate}[label=\roman*.]
\setlength{\itemsep}{0pt}
\item Tail risk should not be allowed in portfolios because unimaginable events happen
\item Options remove the need for precise risk management timing---like paying someone else to be your risk manager
\item Buy protection even when it seems absurd: ``Maybe if someone lights up a nuke in New York, interest rates will be negative 25 percent''
\item Implied volatility is based on historical volatility, but historicals are irrelevant---things can happen for the first time that aren't in any distribution
\item Long volatility positions allow traders to sleep comfortably at night
\end{enumerate}
\end{remark}

\begin{remark} \hlt{Position Sizing Philosophy}\\
Capital preservation is paramount:
\begin{enumerate}[label=\roman*.]
\setlength{\itemsep}{0pt}
\item When uncertain about a market, put a tiny amount of money into it and treat it like a cheap option
\item Great trades come from being fully invested when conviction is high, and minimal when uncertain
\item Disasters and tragedies are often buying opportunities because news-driven panic creates mispricing
\item Changing positions based on news is usually suboptimal---analyze how events will work out before reacting
\end{enumerate}
\end{remark}


% TODO: Add notes from individual manager interviews
% - Dr. Sushil Wadhwani: Central banking perspective
% - Jim Rogers: Pioneer's view on commodities and contrarian investing
% - Dwight Anderson (Ospraie): Commodity specialization
% - Scott Bessent: Equity macro approach
% - Marko Dimitrijević (Everest): Emerging markets expertise

