\subsubsection{Execution Model}

Two basic ways to execute trade: through electronic, or through human intermediary. For electronic execution, achieved through direct market access (DMA), which allows traders to utilise the infrastructure and exchange connectivity of brokerage firms to trade directly on electronic markets.\\
Execution algorithms can be acquired through building, using broker's, or a third-party software vendors.\\
Brokerages offer portfolio bidding, where the 'blind' portfolio for transaction is described by characteristics such as valuation ratios of longs and shorts, sector breakdown, market capitalisation etc. Broker then quote a fee in basis points in terms of the gross market value of portfolio traded. Hence, certainty is provided by the broker to the trader. Once agreement reached, broker receives fee and assumes risk of trading out the portfolio at future market prices, which may be better or worse than prices guaranteed.

\begin{remark} \hlt{Order Execution Algorithm Parameters}
\begin{enumerate}[label=\roman*.]
\setlength{\itemsep}{0pt}
\item Aggressive vs Passive: algorithm make decision of passive vs aggressive order, depending on how immediately the trader wants to do the trade. Market orders are considered aggressive. Limit order at current best order is fairly aggressive, while limit order below current bid is passive.\\
Many exchanges pay providers of liquidity for placing passive orders, charging traders for using liquidity provided. Orders that cross the spread are using liquidity by using a passive order placed by another trader, reducing liquidity available. Paying for liquidity sweetens deal for passive order, only if order is actually executed; passive trader gets better transaction price and a commission rebate from the exchange.\\
Momentum strategies uses more aggressive orders; mean reversion uses more passive orders. A stronger, more certain signal will be executed with greater aggressiveness than a weaker or less certain signal. A middle ground will be to put limit orders between best current bid and offer.
\item Large vs Small Order: a large order may be broken into many smaller orders over a window of time, but risk price moving in adverse direction. Size of chunk depends on transaction cost model estimate, and analysis of correct level of aggressiveness.
\item Hidden vs Visible Order: a queue as a visible order gives away a bit of information. Hidden order will provide no information to the market, staving off imbalances, but reduces priority of trade in the queue.\\
Algorithmic trading utilising hidden order is 'iceberging', which is taking a single larger order and chopping it into many smaller chunks, most posted to order book as hidden orders.
\item Order Routing: if there are several pools of liquidity for the same instrument, smart order routing will be used, which determines which pool of liquidity is most suitable for sending a given order. Depth of liquidity on various ECNs and connectivity speeds are also considered in smart order routing.
\item Cancelling and Replacing Orders: traders may place larger number of orders with no intention of execution, then rapidly cancelling them and replacing them with other orders. This allows gaining of information on how market responds to the changing depth of the book, providing information on how to profit from the pattern of reaction. If trader wants to buy a large number of shares, he may enter a large number of small orders to sell the shares further away from market and cancel, improving market perception.
\end{enumerate}
\end{remark}

\begin{definition} \hlt{High Frequency Trading}\\
Alpha driving strategies on extremely near-term bets (seconds or less) are \hlt{microstructure alphas}, focusing on liquidity patterns in order book. Larger quants may also use this to guide execution models, improving costs of entering trades. Small differences over a single trade add up significantly in the long run.To trade microstructure alpha as independent high frequency strategies, large investments in infrastructure and research must be done.\\
Machine learning techniques may also be used to discern patterns in execution of other player orders. The more inferior the execution models, the easier it is to discern the pattern, allowing the ML strategy to profit from these patterns in the future. Patterns in the shorter timescale are somewhat stable.
\end{definition}

\begin{definition} \hlt{HFT Shark Strategy}\\
Designed to detect large orders that are iceberged, by sending series of very small trades; if each of these small orders get filled quickly, this may be a sign of a large and iceberged order. The shark simply front-run this large, hidden order by placing visible trades in front of the iceberged order. The iceberg strategy must then push prices up to execute trades. When the iceberged order is complete, prices will be pushed up favourably for the shark, which can then exit the position with a quick and relatively riskless profit.
\end{definition}

\begin{remark} \hlt{HFT Trading Infrastructure}\\
Using a broker that act as trading agent allows the infrastructure requirements to be handled by the broker, instead of dealing with the regulatory and other constraints. \\
High frequency strategies may use colocation or sponsored access. Colocation setup is where trader attempts to place trading servers as physically close to the exchange as possible.\\
Financial Information eXchange (FIX) protocol is the choice of real-time electronic communication among users. The software that implements the FIX protocol is free and open source (FIX engine). High frequency traders will likely build their own FIX engines to ensure optimal speeds.
\end{remark}