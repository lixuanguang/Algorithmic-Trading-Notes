\subsubsection{Portfolio Construction Models}

Comes in two major forms: rule-based, optimisers. Rule-based models are based on heuristics, can be exceedingly simple or rather complex, and derived from human experience (trial and error). Optimisers comprises of an objective function and uses algorithms to reach the end goal.

\begin{definition} \hlt{Rule-Based Models}
\begin{enumerate}[label=\roman*.]
\setlength{\itemsep}{0pt}
\item Equal Position Weighting: used if portfolio manager believes that if a position is good enough to own, no other information is needed in determining its size. Strength of signal is not used as input in weighting.\\
Model assumes that there is sufficient statistical strength and power to predict not only direction but also magnitude relative to other forecasts in the portfolio. Portfolio takes few large bets on 'best' forecast, many smaller bets on less dramatic forecasts; may take excess risk in an idiosyncratic event on a seemingly attractive position, resulting in adverse selection bias.
\item Equal Risk Weighting: adjust position sizes inversely to volatilities or a measure of risk. More volatile positions given smaller allocations, less volatile positions given larger allocations.\\
When unit of risk is equalised, it is almost always a backward-looking measurement such as volatility. If volatility changes with time, then model will be misled.
\item Alpha-Driven Weighting: position size based primarily on alpha model. Alpha signal determines size of position, but usually with size limits. Constraints used also includes limits on size of total bet on a group. May also have a function that relates the magnitude of forecast to size of position.\\
If model used in futures trend following, might suffer sharp drawdowns. Reliance on accuracy of alpha.
\item Decision-Tree Weighting: decision path to arrive at the allocation for given instrument, depending on type of alpha model and type of instrument. Constraints may include percentage limits for allocation.\\
Model size grows dramatically if more alpha models or mode types of positions are included.
\end{enumerate}
\end{definition}

\begin{remark} \hlt{Optimisers Models Parameters}\\
Harry Markowitz's mean variance optimisation (MVO) as the pioneer model. Models are based on principles of modern portfolio theory (MPT). Inputs include asset expected return (mean), asset variance, expected correlation matrix. Other inputs include size of portfolio in currency terms, desired risk level (volatility or expected drawdown), and other constraints such as liquidity, universe limits.\\
Model uses an objective function and an algorithm to seek the goal, usually maximising return of portfolio relative to volatility of portfolio returns.
\begin{enumerate}[label=\roman*.]
\setlength{\itemsep}{0pt}
\item Expected Return: alpha models as basis of expected return, which also includes expected direction.
\item Expected Volatility: stochastic volatility forecasting methods is commonly used, as volatility may have high and low periods, with occasional jumps. GARCH model is most used.
\item Expected Correlation: as instrument correlations are not stable over time, it is more appropriate to group assets together before computing correlation within the group.
\end{enumerate}
\end{remark}

\begin{method} \hlt{Optimisation Techniques}
\begin{enumerate}[label=\roman*.]
\setlength{\itemsep}{0pt}
\item Unconstrained Optimisation: most basic form with no constraints. Might provide a single-instrument portfolio, where all money will be invested in instrument with highest risk-adjusted return.
\item Constrained optimisation: constraints include position limits, limits on various groupings of instruments. Might result in constraints driving the portfolio construction more than the optimiser.
\item Black-Litterman Optimisation: blends investor expectations with a degree of confidence about those expectations, and these with historical precedent evident in the data. Adjusts historically observed correlation levels by utilising investor's forecast of return for the various instruments.
\item Grinold and Kahn's Approach: builds a portfolio of signals, instead of sizing positions. To build factor portfolios, each of which are usually rule-based portfolios based on a single type of alpha forecast. Each portfolio backtested, then series of returns are then treated as instruments of a portfolio by the optimiser.\\
Number of factor portfolios is more manageable, usually not more than 20. What is optimised is then a handful of factor portfolios. The model allows for inclusion of risk model, transaction cost model, portfolio size, and risk targets as inputs.
\item Resampled Efficiency: to improve the inputs to optimisation by addressing oversensitivity to estimation error. To resample data using Monte Carlo simulation to reduce estimation error in inputs to the optimiser.
\item Data-Mining Approaches: machine learning techniques such as supervised learning or genetic algorithms used, as MVO involves searching many possible portfolios to find the best.
\end{enumerate}
\end{method}